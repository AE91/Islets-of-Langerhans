 \subsection{Test setup use case 2: Sortering af langerhanske øer}
\begin{center}
		\begin{longtable}{ | m{1.785cm} | m{1.785cm}| m{1.785cm}| m{1.785cm}| m{1.785cm}| m{1.785cm}|m{1.785cm}| } 
			\hline
			\textbf{Krav nr.} &\textbf{ Handling} & \textbf{Forventet resultat} & \textbf{Test-metode} &\textbf{Resultat} & \textbf{ \checkmark \textbackslash -} & \textbf{Initialer og dato} \\ 
			
			\hline
			2.1 &  Kameraet detekterer en Langerhansk ø. & Tælleren for antal sorterede øer stiger & Sorterings-cyklussen er startet. Den Langerhanske ø simuleres vha. simuleringsvæske. Resultatet observeres på GUI.  &  & & \\
			\hline
			
			\hline
			2.2 &  Arduino sender signal til ventilen om åbning. & Udgang til ventilen er høj & Pin D7 måles vha. multimeter. &  & & \\
			\hline
			
			2.3 &  Ventilen åbner  & Ventilen er åben & Observeres ved at se på ventilen.  &  & & \\
			\hline
			
			2.4 &  Arduino sender signal til ventilen om lukning.  & Udgang til ventilen er lav & Pin D7 måles vha. multimeter.  &  & & \\
			\hline
			
			2.5 &  Ventilen lukker. & Ventilen er lukket & Observeres ved at se på ventilen. &  & & \\
			\hline
			
			
		\end{longtable}
		
	\end{center}
	\pagebreak