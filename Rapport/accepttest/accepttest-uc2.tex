\newpage
 \subsection{Use Case 2: Sortering af langerhanske øer}

	\begin{center}
		\begin{longtable}{ | m{4cm}| m{8.5cm}|} 
			\hline
			\textbf{Krav nr.} & 2.1  \\ 
			\hline
			\textbf{Handling} & Systemet detekterer en Langerhansk ø.  \\
			\hline
			\textbf{Forventet resultat} & Tælleren for antal sorterede øer stiger \\
			\hline
			\textbf{Testmetode}  & Sorteringscyklussen er startet. Det observeres på GUI, at tælleren for antal detekterede øer stiger.   \\
			\hline
			\textbf{Resultat}  &    \\
			\hline
			\textbf{Angiv godkendelse} &     \\
			\hline
			\textbf{Initialer} &     \\
			\hline
			\textbf{Dato} &    \\
			\hline
		\end{longtable}
	\end{center}	
	
	\begin{center}
		\begin{longtable}{ | m{4cm}| m{8.5cm}|} 
			\hline
			\textbf{Krav nr.} & 2.2  \\ 
			\hline
			\textbf{Handling} & Arduino sender signal til ventilen om åbning.  \\
			\hline
			\textbf{Forventet resultat} & Udgang til ventilen er høj (5V) \\
			\hline
			\textbf{Testmetode}  & Pin D8 måles vha. et multimeter med reference til GND på Arduino.  \\
			\hline
			\textbf{Resultat}  &    \\
			\hline
			\textbf{Angiv godkendelse} &     \\
			\hline
			\textbf{Initialer} &     \\
			\hline
			\textbf{Dato} &    \\
			\hline
		\end{longtable}
	\end{center}	
	
	\begin{center}
		\begin{longtable}{ | m{4cm}| m{8.5cm}|} 
			\hline
			\textbf{Krav nr.} & 2.3  \\ 
			\hline
			\textbf{Handling} & Ventilen åbner   \\
			\hline
			\textbf{Forventet resultat} & Ventilen er åben \\
			\hline
			\textbf{Testmetode}  & Observeres ved at se, at der løber væske ned i beholderen med isolerede øer  \\
			\hline
			\textbf{Resultat}  &    \\
			\hline
			\textbf{Angiv godkendelse} &     \\
			\hline
			\textbf{Initialer} &     \\
			\hline
			\textbf{Dato} &    \\
			\hline
		\end{longtable}
	\end{center}
	
\newpage	
	
	\begin{center}
		\begin{longtable}{ | m{4cm}| m{8.5cm}|} 
			\hline
			\textbf{Krav nr.} & 2.4  \\ 
			\hline
			\textbf{Handling} & Arduino sender signal til ventilen om lukning.   \\
			\hline
			\textbf{Forventet resultat} & Udgang til ventilen er lav(0V) \\
			\hline
			\textbf{Testmetode}  & Pin D8 måles vha. et multimeter med reference til GND på Arduino.  \\
			\hline
			\textbf{Resultat}  &    \\
			\hline
			\textbf{Angiv godkendelse} &     \\
			\hline
			\textbf{Initialer} &     \\
			\hline
			\textbf{Dato} &    \\
			\hline
		\end{longtable}
	\end{center}	
	
	\begin{center}
		\begin{longtable}{ | m{4cm}| m{8.5cm}|} 
			\hline
			\textbf{Krav nr.} & 2.5  \\ 
			\hline
			\textbf{Handling} & Ventilen lukker. \\
			\hline
			\textbf{Forventet resultat} & Ventilen er lukket \\
			\hline
			\textbf{Testmetode}  & Observeres ved at se væsken løber ud i wastebeholderen  \\
			\hline
			\textbf{Resultat}  &    \\
			\hline
			\textbf{Angiv godkendelse} &     \\
			\hline
			\textbf{Initialer} &     \\
			\hline
			\textbf{Dato} &    \\
			\hline
		\end{longtable}
	\end{center}