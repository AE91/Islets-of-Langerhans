  % \section{Accepttest af ikke funktionelle krav}
\begin{center}
		\begin{longtable}{ | m{1.785cm} | m{1.785cm}| m{1.785cm}| m{1.785cm}| m{1.785cm}| m{1.785cm}|m{1.785cm}| } 
			\hline
			\textbf{Krav nr.} &\textbf{ Kvalitetskrav} & \textbf{Forventet resultat} & \textbf{Test-metode} &\textbf{Resultat} & \textbf{ \checkmark \textbackslash -} & \textbf{Initialer og dato} \\ 
			
			\hline
			1 &  Hastigheden på systemet skal være Højere end 30 øer sorteret pr. minut & Normalforløbet ved en sorteringscyklus følges (Use Case 2: Sortering af Langerhanske Øer, s. \pageref{uc:2}), hvor der måles med et stopur. Stopuret stoppes efter sorteringsprocessen er færdig, derefter regnes hastighed ud ved 
$\frac{Antal øer}{minutter}$			
			
			 & Når sorteringcyklussen er færdig er
			$\frac{Antal øer}{minutter}>30$
 &  & & \\
			\hline
			
			2 &  1 Mere end 90\% af de isolerede øer skal være faktiske øer 
(Sandt pos: > 90\%)
 & Normalforløbet ved en sorteringscyklus følges (Use Case 2: Sortering af Langerhanske Øer, s. \pageref{uc:2}). Efter endt cyklus, skal en kyndig person tælle antallet af faktiske øer. Dette holdes op i mod antallet af isoleret øer. & Når sorteringcyklussen er færdig er

 $\frac{antal talte øer}{antal isoleret}*100=>90$


.   &  & & \\
			\hline
			
			2 &  2 Der skal være mindre end 5\% af de isolerede øer, der ikke er øer
(Falsk pos: < 5\%)
 & Normalforløbet ved en sorteringscyklus følges (Use Case 2: Sortering af Langerhanske Øer, s.  \pageref{uc:2}). Efter endt cyklus, skal en kyndig person tælle antallet af faktiske øer. Dette holdes op i mod antallet af isoleret øer.  & Når sortering-cyklussen er færdig er

$\frac{antal talte øer}{antal isoleret}*100=>95$


.   &  & & \\

			\hline
2 &  3 Der skal være mindre end 5\% af øerne i opløsningen der ikke er blevet isoleret
(Falsk neg: < 5\%)

 & Normalforløbet ved en sorteringscyklus følges (Use Case 2: Sortering af Langerhanske Øer, s.  \pageref{uc:2}). Efter endt cyklus, skal en kyndig person tælle antallet af øer der er isoleret og antallet af øer i waste beholderen.  & Når sortering-cyklussen er færdig er

 $\frac{antal talte øer i waste}{antal talte isoleret øer}*100= >95$

.   &  & & \\
			\hline		
			
			3 &  over 90\% af det oprindelige antal, skal være isoleret.

   & En opløsning med et kendt antal øer benyttes. Hvor efter normalforløbet ved en sorteringscyklus følges (Use Case 2: Sortering af Langerhanske Øer, s. \pageref{uc:2}). Efter endt cyklus, skal en kyndig person tælle antallet af øer der er isoleret. Dette antal holdes op i mod antallet af øer i opløsningen fra start. & Når sortering-cyklussen er færdig er
 %\begin{align}
 $\frac{antal talte øer i opløsningen fra start}{antal talte isoleret øer}*100= >90$
 %\end{align}
.   &  & & \\
			\hline		
			
4 &  over 90\% af det oprindelige antal, skal være genkendt
   & En opløsning med et kendt antal øer benyttes. Hvor efter normalforløbet ved en sorteringscyklus følges (Use Case 2: Sortering af Langerhanske Øer, s. \pageref{uc:2}). Efter færdig endt cyklus, holdes antal af detekteres op i mod det oprindelige antal af øer fra starten.
    & Når sortering-cyklussen er færdig er

 $$\frac{antal detekteret øer}{antal oprindelige øer}*100= >90$$

.   &  & & \\
			\hline		
			
			5 &  Systemet skal kunne sortere øer, der har en størrelse mellem  \SI{100}{\micro\metre}  og  \SI{300}{\micro\metre} 
   & En opløsning med en østørrelse på \SI{100}{\micro\metre}  og \SI{300}{\micro\metre} benyttes  Hvor efter normalforløbet ved en sorteringscyklus følges (Use Case 2: Sortering af Langerhanske Øer, s. \pageref{uc:2}). Efter endt cyklus observeres det om systemet har sorteret de specificerede størrelser.
    & Begge ø størrelser er isoleret
    &  & & \\
			\hline			
			
			6 &  Systemet skal kunne give informationer omkring opløsningens øer, både størrelse og form. 
   & Normal-forløbet ved en sorteringscyklus følges. 
    & Efter endt cyklus, skal data filen kontrolleres om den har de specificerede værdier.
    &  & & \\
			\hline	
			
		\end{longtable}
		
	\end{center}
	\pagebreak