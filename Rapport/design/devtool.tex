\section{Udviklingsværktøjer}
\subsection{MATLAB}
Selve softwaren til systemet udvikles i MATLAB. Versionen der er anvendt er R2015b. Af tilføjelsespakker bruges følgenede:
\begin{itemize}
\item Arduino Support Package (14.2.2)
\item Webcam Support Package (15.2)
\end{itemize}
Disse pakker anvendes til, at interagere med Arduinoen og USB mikroskopet.
\subsection{GitHub}
Til versionsstyring af projektdokumentationen og source kode anvendes GitHub, som bygger på open source versions styrings systemet Git. Her opdateres der løbende ændringer, så det nyeste dokumentation og source kode altid er tilgængeligt. 
Som user interface til GitHub anvendes GitHub Desktop .
\subsection{Pivotal Tracker}
Til projektstyring anvendes Pivotaltracker, som er et online værktøj baseret på SCRUM. I Pivotaltracker defineres projektets arbejdsopgaver, hvorefter de tildeles point alt efter hvor stor arbejdsbyrden er. De enkelte opgaver prioriteres herefter i projektets backlog, hvor Pivotaltracker automatisk tilføjer opgaver til den igangværende sprint udfra den nuværende “velocity”.

Det betyder, at der er fuldstændig styr på om projektet går for langsomt, eller om udviklingen af projektet er godt med. Dette kan sammenlignes med gatestate arbejdsmetoden, hvor der er flere deadlines end der måske vil være i for eksempel vandfaldsmetoden. 