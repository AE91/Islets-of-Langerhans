\newpage
\subsubsection{cameraFeed} \label{subsub:camfeed}
Denne funktion er implementeret udfra beskrivelsen i designdokumentet (\ref{subsub:Dcamerafeed} s. \pageref{subsub:Dcamerafeed}). Dog er den ændret i forhold til ikke at modtage et feed fra kameraet, men i stedet indlæses et billede fra en prædefineret mappe. 
Funktionen har handles som input og output. 

I et while loop itereres der gennem mappen, hvor en counter variabel definerer hvilket billede der skal indlæses. Efter dette nedskaleres billedet til halv størrelse med \textit{imresize}. Dette gøres for, at optimere tiden det tager at behandle billedet, samt vise det på GUI. I denne forbindelse kunne der være en potentiel risiko for, at information gik tabt på billedet i nedskalleringen. Dette blev testet på en række billeder for, at se om algoritmen (beskrevet i \ref{subsub:IMdetectIslets} ville fejle på det nedskallerede billede, samt om tiden for billedebehandlingen blev forbedret. \fxnote{Mangler dokumentation i form af billeder og tic toc} Billedet gives herefter som parameter til funktionen detectIslet (\ref{subsub:IMdetectIslets}). %I while loopet kaldes der til sidst pause(0,1), som pauser while loopet i 0,1 sekund. Dette simulerer at kameraet optager med 10 f/s. 

Matlab implementeringen er vedlagt i bilag \ref{bilag:cameraFeed}.