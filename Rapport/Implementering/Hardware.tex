  \section{Hardware}
Efter den indledende identifikation og specifikation i design og kravspecifikationen kan implementeringsfasen nu udføres. Hardware implementering af \textit{The Cell Collector} består af enhedstest af hver komponent, med følgende dokumentation. Enhedstestene er lavet i den skrevne rækkefølge og integreret i samme. I dette afsnit vil der være kredsløbsdiagrammer, teori, beregninger og beskrivelser, hvilket er udarbejdet sideløbende med udviklingen af produktet. 
 	
 \subsection{Vægtcelle}
Vægtcellen skal kontroller om opløsningsbeholderen er tom. Dette afsnit er skrevet i tæt efterfølge af design dokumentet.

\subsubsection{Enhedstest af vægtcelle }

Efter at diagrammet er fastlagt, testes forbindelserne nu på et \textit{fumlebræt} se figur \ref{fig:loadcelltest} for test opstilling

  \begin{figure}[H]
	\centering
	\includegraphics[width=0.9\textwidth]{billeder/Hardware/diagrammer/Drawing1.jpg}
	\caption{Test opstilling for vægtcelle}
	\label{fig:loadcelltest}
\end{figure}
De primære dele af test koden har været  \textit{sensorValue = analogRead(A0);} og \textit{Serial.println(sensorValue);}, hvor ved at inputtet på \textit{A0} er læst i \textit{serial monitor} som vist på figur \ref{fig:loadcell_test} ved langsom tømning af beholderen.

\begin{figure}[H]
	\centering
	\includegraphics[width=0.9\textwidth]{billeder/Hardware/diagrammer/loadcellunittestbits.JPG}
	\caption{Værdier fra A0 i \textit{serial monitor}}
	\label{fig:loadcell_test}
\end{figure}

 Se bilag \ref{bilag:TKloadcell} for at se hele koden til testen. Til enhedstesten er der brugt et voltmeter til, at måle udgangsspændingen på INA114 for, at se om arduinoens ADC læste rigtigt. Til at sammenligne med voltmeteret, blev \ref{eq:trintilvolt} brugt til at konvertere ADC'ens bits værdi om til spænding.
 
 \begin{align}
 analogRead(A_0)*\frac{5}{1024}=\text{spænding i volt}
 \label{eq:trintilvolt}
 \end{align}
Testopstilling af vægtcellen ser ud som på \ref{fig:loadcell_mont} med celleopløsningsbeholderen. I softwaren kræves det en kalibrering for at vægtcellen er præcis, dette er implementeret i afsnit \ref{subsub:softwareloadcell}.
 
 \begin{figure}[H]
	\centering
	\includegraphics[width=0.5\textwidth]{billeder/Hardware/diagrammer/loadcell_montering.pdf}
	\caption{Illustration af opstilling med vægtcelle og celleopløsningsbeholder}
	\label{fig:loadcell_mont}
\end{figure}

\subsection{Pumpe}
 Pumpen består af en DC motor og ud fra databladet kan det ses at den skal bruge 12VDC og 0.3A for at køre. Da arduinoen langt fra kan leverer den nødvendige strøm til at drive motoren, skal der bruges en motordriver med en ekstern strømforsyning.
 

\subsubsection{Enhedstest af motor og motordriver}
\label{subsubsec:enhedstestmotor}
På figur \ref{fig:Motorbreadboard} kan motordriveren og arduinoens opstilling ses på et \textit{fumlebræt}. I opstilling er der to lysindikatorer, som viser om motoren kører den ene eller den anden vej. Retningen kan styres ved at bytte om på hvilken udgang der sættes lav(0V) og hvilke der PWM konfigureres.
 

 \begin{figure}[H]
	\centering
	\includegraphics[width=1\textwidth]{billeder/Hardware/diagrammer/Motorbreadboard.JPG}
	\caption{Illustration af opstilling med motor(pumpe), motordriver og arduino}
	\label{fig:Motorbreadboard}
\end{figure}

For at teste motordriverens evne til at lave PWM med 12 volt, er PWM signalet fra arduinoen og motordriveren sammenlignet med hinanden vha. et oscilloskop. Billeder fra denne test er vist på  billederne \ref{fig:25PWM}, \ref{fig:50PWM}, \ref{fig:75PWM} og \ref{fig:100PWM} den høje gule graf er fra motordriveren og den grønne lave graf er fra arduinoen.

\newpage

 \begin{figure}[htbp] \centering
\begin{minipage}[b]{0.48\textwidth} \centering
\includegraphics[width=1.00\textwidth]{billeder/Hardware/motor25PWM.jpg} % Left picture
\end{minipage} \hfill
\begin{minipage}[b]{0.48\textwidth} \centering
\includegraphics[width=1.00\textwidth]{billeder/Hardware/motor50PWM.jpg} % Right picture
\end{minipage} \\ % Captions og labels
\begin{minipage}[t]{0.48\textwidth}
\caption{Arduino og motordriver med 25$\%$ PWM} % Left caption and label
\label{fig:25PWM}
\end{minipage} \hfill
\begin{minipage}[t]{0.48\textwidth}
\caption{Arduino og motordriver med 50$\%$ PWM} % Right caption and label
\label{fig:50PWM}
\end{minipage}
\end{figure}

 \begin{figure}[htbp] \centering
\begin{minipage}[b]{0.48\textwidth} \centering
\includegraphics[width=1.00\textwidth]{billeder/Hardware/motor75PWM.jpg} % Left picture
\end{minipage} \hfill
\begin{minipage}[b]{0.48\textwidth} \centering
\includegraphics[width=1.00\textwidth]{billeder/Hardware/motor100PWM.jpg} % Right picture
\end{minipage} \\ % Captions og labels
\begin{minipage}[t]{0.48\textwidth}
\caption{Arduino og motordriver med 75$\%$ PWM} % Left caption and label
\label{fig:75PWM}
\end{minipage} \hfill
\begin{minipage}[t]{0.48\textwidth}
\caption{Arduino og motordriver med 100$\%$ PWM} % Right caption and label
\label{fig:100PWM}
\end{minipage}
\end{figure}

De elementære funktioner, som er brugt til test koden er \textit{digitalWrite(9, LOW);} og \textit{analogWrite(10, 127);}. se hele testkoden i bilag \ref{bilag:TKpumpe}

\newpage

 \subsection{Ventil}
Ventilen skal sortere de langerhanske øer fra det eksokrine væv, derfor er ventilen en vigtig del af hardwaren. I integrationstesten bør tidsintervallet med hvornår ventilen skal åbnes og lukkes udføres, for en sikring af at øerne bliver isoleret.
\subsubsection{Enhedstest for ventil}
På figur \ref{fig:ventilbreadboard} er det illustreret hvordan testopstilling er sat op på et \textit{fumlebræt}. I test koden er der hovedsagligt brugt \textit{digitalWrite(8, LOW);} og \textit{digitalWrite(8, HIGH);}, hvorved arduinoen sætter udgangen lav(0V) og høj(5V). Motordriveren bruger signalet til det at give ventilen 12V når den er høj. Se bilag \ref{bilag:TKventil} for hele testkoden

\begin{figure}[H]
	\centering
	\includegraphics[width=1\textwidth]{billeder/Hardware/diagrammer/Ventilbreadboard.JPG}
	\caption{Illustration af testopstilling med ventil og motordriver}
	\label{fig:ventilbreadboard}
\end{figure} 
 
\newpage
 \subsection{Kameralys}
 \label{subsec:kameralys}
Kameralyset skal hjælpe operatøren med at optimere lysforholdet til kameraet for, at maksimerer forholdene og muligheden for at detektere de langerhanske øer.  

\subsubsection{Enhedstest for Kameralyset}
På figur \ref{fig:LEDbreadboard} vises testopstilling på et \textit{fumlebræt} for Kameralyset er der brugt samme test kode som til pumpen se afsnit \ref{subsubsec:enhedstestmotor}. 

\begin{figure}[H]
	\centering
	\includegraphics[width=1\textwidth]{billeder/Hardware/diagrammer/LEDbreadboard.JPG}
	\caption{Illustration af testopstilling med kameralys og motordriver}
	\label{fig:LEDbreadboard}
\end{figure}

 
\newpage 
\subsection{Ikke-elektroniske dele}
På figur \ref{fig:nonelectronic} vises en illustration af de ikke-elektroniske dele for hele systemet. De ikke-elektroniske dele består af en opløsningsningsbeholder, som navnet fortæller indeholder opløsningen med de langerhanske øer. Fra opløsningsbeholder går der en teflonslange, den er hård hvilket en fordel når den skal suge væsken op. For at en peristaltisk pumpe kan benytte skal det være en silikoneslange, fordi den er eftergivelig. For at komme fra teflonslange til silikoneslange er der brugt en adapter. Derefter føres silikone slangen til pumpen til ventilen, hvor fra der går en silikoneslange fra hver af ventilens studser. Fra ventilens studser går en silikone slange til wastebeholderen og beholderen til de isoleret øer.

\begin{figure}[H]
	\centering
	\includegraphics[width=1\textwidth]{billeder/Hardware/ikkeelektronisk.pdf}
	\caption{Illustration af ikke elektroniske dele}
	\label{fig:nonelectronic}
\end{figure}

Til test af ikke-elektroniske dele, er simuleringsvæsken med timianfrø og demineraliseret vand brugt. Timianfrøene der er brugt har en størrelse på 0.4mm hvilket burde komme igennem, men frøene stopper ved ventilen. På grund af at frøene stopper ved ventilen ses det som en fejlet test, hvilket bør undersøges. I ventilens datablad \ref{bilag:161T031} står der, at ventilen har tilgange på 1mm. 

\newpage
\subsubsection{Stykliste} 
\begin{center}
		\begin{longtable}{ | m{6cm} | m{4cm}| m{2cm}| } 
			\hline
			\textbf{Beskrivelse} &\textbf{Type} & \textbf{Antal} \\ 
			\hline
			Opløsningsbeholder & ML 33184 & 1 stk \\ 
			\hline
			Flaskeadapter låg & ML 75441 & 1 stk \\ 
			\hline
			1/4" Prop til låg & UP P309 & 1 stk \\ 
			\hline
			1/16" Skrue flangeløs & UP P201 & 1 stk \\ 
			\hline
			1/16" Ferrule flangeløs & UP P200 & 1 stk \\ 
			\hline
			Flaskeadapter filter & n/a & 1 stk \\ 
			\hline	
			Teflonslange & ML 94142 & 0.2 m \\ 
			\hline
			Slangeadapter & UP P630 + UP646 & 1 stk \\ 
			\hline
			Silikoneslange(0.5mm) & n/a & 0.3 m \\ 
			\hline
			Slangestudser til ventil(10-32) & VP X210-60005 & 3stk \\ 
			\hline
		\end{longtable}
\end{center}

 
 
 
 
\subsection{PCB design} 
For at samle hardwaren i projektet er der lavet et printkort formet, som et shield til arduino platformen. På figur \fxnote{beregninger osv.}

\subsubsection{PCB kredsløbsdiagram}

\begin{figure}[H]
	\centering
	\includegraphics[width=1\textwidth]{billeder/hardware/Diagram.png}
	\caption{Diagram for printkortet}
	\label{fig:PCBdiagram}
\end{figure}

\subsubsection{PCB toplayout}

\begin{figure}[H]
	\centering
	\includegraphics[width=1\textwidth]{billeder/hardware/Toplayout.png}
	\caption{PCB toplayout}
	\label{fig:PCBtoplayout}
\end{figure}

\subsubsection{PCB bundlayout}

\begin{figure}[H]
	\centering
	\includegraphics[width=1\textwidth]{billeder/hardware/Bundlayout.png}
	\caption{PCB bundlayout med groundplane}
	\label{fig:PCBbundlayout}
\end{figure}

\subsubsection{Stykliste til PCB: 30-77-1}
Til samling af printet skal der bruges komponenterne i nedenstående tabel. 
\begin{center}
		\begin{longtable}{ | m{6cm} | m{4cm}| m{2cm}| } 
			\hline
			\textbf{Navn} &\textbf{Type} & \textbf{Antal} \\ 
			\hline
			C2,C4,C5 & 10nF & 3 stk \\ 
			\hline
			C1, C3 & 100nF & 2 stk \\ 
			\hline
			IC1 & INA114 & 1 stk \\ 
			\hline
			IC2 & L293D & 1 stk \\ 
			\hline
			Jp1-pinheader(stackable) & 1x10 & 1 stk \\ 
			\hline
			Jp2,3-pinheader(stackable) & 1x8 & 2 stk \\ 
			\hline
			Jp4-pinheader(stackable) & 1x6 & 1 stk \\ 
			\hline
			LED1-Green & 5mm(T13/4) & 1 stk \\ 
			\hline	
			LED2-Yellow & 5mm(T13/4) & 1 stk \\ 
			\hline
			LED3-Blue & 5mm(T13/4) & 1 stk \\ 
			\hline
			R1-R7 & 440$\Omega$ & 7 stk \\ 
			\hline
			RG & 51$\Omega$ & 1 stk \\ 
			\hline
			X1,X3 & molex 5566-4 & 2 stk \\ 
			\hline
			X2 & molex 5268-04 & 1 stk \\ 
			\hline
			X4 & molex 5268-05 & 1 stk \\ 
			\hline
		\end{longtable}
\end{center}

\subsubsection{Stykliste stik til print}
For at sætte pumpe, ventil, vægtcelle og kameralyset til printkortet er komponenterne i nedenstående tabel nødvendige. 
\begin{center}
		\begin{longtable}{ | m{6cm} | m{4cm}| m{2cm}| } 
			\hline
			\textbf{Navn} &\textbf{Type} & \textbf{Antal} \\ 
			\hline
			X1,X3 & 2x2 Minifit Jr.(5557) & 2 stk \\ 
			\hline
			X1,X3 & Crimp(5556) & 4 stk \\ 
			\hline
			X2 & Polantal 4(5037) & 1 stk \\ 
			\hline
			X4 & Polantal 5(5037) & 1 stk \\ 
			\hline
			X2,X4 & Crimp(5263) & 9 stk \\ 
			\hline
		\end{longtable}
\end{center}

beregninger her til
 \fxnote{mangler}
 