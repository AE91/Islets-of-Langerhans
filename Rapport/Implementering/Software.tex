\section{Software}
Software implementeringen af \textit{The Cell Collector} beskrives det i detaljer, hvordan de enkelte Matlab funktioner er implementeret. Herudover vil der i afsnittet være beskrivelser og resultater af enhedstest. 
 
\subsection{Kamera}
\subsubsection{Test}
Det indkøbte kamera er testet ved at tage en serie af billeder af langerhanske øer. Billedserien skulle i første omgang danne grundlag for den videre billedbehandling. Første test bestod af 107 billeder taget af langerhanske øer, herunder billeder kun af isolerede øer og enkelte baggrundsbilleder. Forsøgsopstillingen var en efterligning af den nuværende sorteringsproces, hvor opløsningen hældes i petriskåle. Petriskålen placeres på en sort bagplade, hvorefter operatøren isolerer øerne ved at kigge gennem et mikroskop. I forsøgsopstillingen anvendtes der i stedet for mikroskopet det indkøbte kamera. En række billeder blev udvalgt til yderligere analyse, hvor Søren Gregersen udpegede hvilke elementer der var øer. På billede \ref{fig:isletSG} er der markeret, hvor en ø er placeret.

\begin{figure}[H]
	\centering
	\includegraphics[width=0.6\textwidth]{billeder/software/sgbillede.png}
	\caption{Øer udpeget af Søren Gregersen}
	\label{fig:isletSG}
\end{figure}

Efter nærmere analyse af billederne viste det sig at det indkøbte kamera ikke var tilstrækkelig kvalitet til at lave segmentering på billederne. Når øerne blev observeret gennem et almindeligt mikroskop var der langt større kontrast i mellem øerne og det ekstra væv. På billederne fra kameraet er der ikke denne forskel, hvilket udelukker segmentering på denne parameter. Det kan ses på billede \ref{fig:isletSG} at der er områder hvor lysintensiteten er lige så høj, som de steder hvor øerne er markeret. Dette gør, at segmentering på baggrund af lysintensiteten kan udelukkes. Herudover er det tydeligt, at billederne ikke er skarpe nok, hvilket gør at det er svært at vurdere hvad der er øer og ikke øer. De parametre, hvor kameraets kvalitet ikke er tilstrækkelig er derfor bl.a. autoeksponering og autofokus. En bedre styring af autoeksponeringen ville kunne give et bedre kontrast forhold i mellem øerne og det omkringliggende væv, mens en bedre autofokus ville hjælpe på skarpheden i billedet, hvilket muliggøre segmentering baseret på strukturer.

Herudover er der er en polariserende effekt på billederne, hvor nogle af elementerne lyser meget kraftigt op grundet belysningen. Dette gør det er svært at vurdere strukturer og størrelser på elementerne i billedet. 

Efter den første test blev det i samarbejde med vejleder og kunde besluttet, at det indkøbte kamera ikke er af tilstrækkelig kvalitet til at detektere langerhanske øer. I dette system bliver der i stedet udviklet et sæt af billeder, som skal simulerer flowet i slangerne. Billederne skal indeholde langerhanske øer, ekstra væv og tilfældig støj for at få dem til at ligne tæt på det man observerer gennem et almindeligt mikroskop. Udfra disse billeder skal langerhanske øer detekteres, hvorefter systemet skal åbne ventilen. Billedesættet skal ses som en simulering af kameraet. I en senere prototype vil et kamera af højere kvalitet være påkrævet. Her er det især krav til bedre autoeksponering og autofokus som er essentielle. Dette vil være nødvendigt for at øge kontrasten i mellem øerne og det omkringliggende væv for bedre segmentering. Herudover vil et kamera med polariseringsfilter være en mulighed til at fjerne evt. genskær fra belysningen. Hos Farnell er der andre producenter end det indkøbte Duratool mikroskop, som bl.a. har polariseringsfilter. En test af forskellige kameraer ville være oplagt til finde det ideelle kamera til optagelse af langerhanske øer. 
\subsubsection{Simulering af kamera}
Til at generere et billedsæt, der simulerer langerhanske øer, er der udviklet et Matlab script. Scriptet består overordnet af 3 faser. Den første fase består i at segmentere langerhanske øer udfra et billede og oprette dem som en maske. I anden fase laves en maske bestående af ekstra væv, mens der i tredje fase simuleres flow. I den sidste fase gemmes også de enkelte billeder i formatet .png (Portable Network Graphic). De enkelte faser er nærmere beskrevet under deres afsnit. Der anvendes 3 billeder til grundlag for generingen. Det ene billede viser isolerede øer. Det andet billede viser opløsningen indeholdende øer og ekstra væv. Det sidste billede er et baggrundsbillede uden øer eller ekstra væv. Dette billede anvendes til baggrunden til de genererede billeder. De 3 billeder er herunder vist.

\begin{figure}[htbp] \centering
\begin{minipage}[b]{0.3\textwidth} \centering
\includegraphics[width=1.00\textwidth]{billeder/software/1.jpg} % Left picture
\end{minipage} \hfill
\begin{minipage}[b]{0.3\textwidth} \centering
\includegraphics[width=1.00\textwidth]{billeder/software/2.jpg} % Right picture
\end{minipage} 
\hfill
\begin{minipage}[b]{0.3\textwidth} \centering
\includegraphics[width=1.00\textwidth]{billeder/software/3.jpg} % Right picture
\end{minipage} \\  % Captions og labels
\begin{minipage}[t]{0.3\textwidth}
\caption{Billede indeholdende langerhanske øer} % Left caption and label
\label{fig:img1}
\end{minipage} \hfill
\begin{minipage}[t]{0.3\textwidth}
\caption{Billede indeholdende ekstra væv} % Right caption and label
\label{fig:img2}
\end{minipage}
\hfill
\begin{minipage}[t]{0.3\textwidth}
\caption{Baggrundsbillede} % Right caption and label
\label{fig:img3}
\end{minipage}
\end{figure}

\textbf{Fase 1}

Segmenteringen af langerhanske øer sker ud fra billede 1 \ref{fig:img1}, hvor funktionen circleFinder anvendes til at finde centrum og radius på af de fundne celler. 

\textbf{Fase 2}

I anden fase bliver det ekstra væv segmentet ud fra billede 2 (REF). Til dette er der anvendt Color Threshold appen i Matlab. Ved hjælp af denne app er der lavet en funktion, som opretter en logisk maske af det ekstra væv. Opsætningen i appen er vist i figur (REF) \fxnote {Indsæt figur}. Yderligere er der anvendt morfologiske operationer til at fjerne uønskede objekter fra masken, samt fjerne støj. Til at fjerne uønskede objekter er funktionen bwareafilt anvendt, med parametrene 150 og 500. Dette fjerner alle objekter under 150 og over 500 sammenhørende pixels. Til at udfylde huller i de enkelte objekter er funktionen imfill brugt. 

\textbf{Fase 3}

I fase 3 sker selve flow simuleringen. Flowsimuleringen er opbygget på den måde, at den består af henholdvis en sekvens indeholdende en langerhansk ø efterfulgt af en sekvens uden en ø. I selve programmet indlæses et nyt billede hvert 0,1 sekund. Derfor skal der generes en passende mængde billeder, som programmet kan indlæse. Fase 3 er implementeret så der minimum genereres 252 eller maksimalt 432 billeder, hvilket giver en samlet sekvenslængde på 25,2 eller 43,2 sekund. Grunden til at antallet af billeder varierer er at længden af sekvensen uden en langerhansk ø bestemmes udfra en random variabel. I scriptet genereres der i alt 18 sekvenser. Det betyder, at der passerer i mellem 25 og 43 øer i minuttet.
\begin{align}
\frac{18}{n/10} * 60 = \text{Antal øer pr. minut}
\end{align}
I figur \ref{fig:boxplot} er vist et boxplot som viser distributionen af hvor mange øer der passere i minuttet. Det ses at medianen ligger over 30 øer pr. minut, hvilket betyder at der i gennemsnit vil komme over 30 øer pr minut.

 \begin{figure}[H]
	\centering
	\includegraphics[width=0.5\textwidth]{billeder/software/boxplot.png}
	\caption{Boxplot af distrubutionen af øer pr. minut}
	\label{fig:boxplot}
\end{figure}

Selve flowsimuleringen sker i et for loop. Først udvælges en tilfældig celle ved hjælp af randi (uniform fordelt random variable). I for loopet opdateres dens center position ud fra 2 variabler (newXPos og newYPos). X positionen springer med et fast interval for hver iteration (160 px). Inden for loopet fastsættes start positionen for cellen med randi, som giver et tal mellem 0 og 1200 px, som er højden på billedet. Herefter opdateres den nye Y position ved hjælp af randn (normal fordelt random variabel) med middelværdi sat til startpositionen og en standard afvigelse på 50 pixels. I figur \ref{fig:flowsim} er flow simulationen illustreret for de i alt 18 sekvenser, med en graf for hver celle. Det ses at cellen flytter sin Y position for hver iteration i for loopet.

\begin{figure}[H]
	\centering
	\includegraphics[width=0.5\textwidth]{billeder/software/Simulation.png}
	\caption{Illustration over flow simuleringen}
	\label{fig:flowsim}
\end{figure}



\newpage
\subsection{Program flow}
Flowchartet i figur \ref{fig:softwareFlowchart} viser flowet i det implementerede Matlab program. De enkelte blokke indikerer de udviklede Matlab funktioner, mens beslutningsknuderne indikerer knaptryk, om beholderen eller om en ø er detekteret. 
\begin{figure}[H]
	\centering
	\includegraphics[width=0.6\textwidth]{billeder/software/software_flowchart-crop.pdf}
	\caption{Flowchart for programmet}
	\label{fig:softwareFlowchart}
\end{figure}

\newpage
\subsection{Data struktur} \label{software_datastruktur}
Dette afsnit beskriver data strukturen for det udviklede Matlab program. Når man arbejder med GUI i Matlab anvendes structed handles, som indeholder alle UI Controls såsom knapper, tekstfelter og axes. Figur \ref{fig:handles} viser strukturen af handles. Udover UI controls indeholder handles structs med konstanter og data omkring de detekterede øer. 
\begin{figure}[H]
	\centering
	\includegraphics[width=1\textwidth]{billeder/software/handles-crop.pdf}
	\caption{Data struktur}
	\label{fig:handles}
\end{figure}

\newpage
\subsection{Matlab funktioner}
I dette afsnit er implementeringen af de enkelte funktioner i Matlab nærmere beskrevet. Selve implementeringen tager udgangspunkt i beskrivelserne fra design dokumentet. %Herudover er der beskrevet test af de enkelte funktioner.  


\subsubsection{initArduino}
Implementering af Arduino
I denne funktion opsættes og initialiseres Arduinoen. Til dette anvendes funktionen \textit{Arduino}, fra Arduino Support pakken. Som parameter til funktionen angives port navnet og board typen (i dette system en Arduino Uno). Arduino variablen gemmes herefter i handles, som en variabel.

Herudover opsættes Arduinoens input og output i funktionen. \fxnote{Mangler beskrivelse}

Selve initialiseringen sker i et try/catch statement, hvis Arduinoen ikke kan initialiseres åbnes en dialogboks (figur: \ref{fig:initArduino}), hvor operatøren kan vælge om systemet skal forsøge at oprette forbindelse igen. 

\begin{figure}[H]
	\centering
	\includegraphics[width=0.5\textwidth]{billeder/software/initArduino.png}
	\caption{Dialogboks til systembesked}
	\label{fig:initArduino}
\end{figure}

Implementeringen af initArduino er vist i nedenstående kode.
\begin{lstlisting} 
while true
    % Try/Catch statement
    try
        % Initialize Arduino and save variable in handles
        handles.a = arduino('/dev/cu.usbmodem1411','Uno');
        % If succes - break while loop
        break
    catch
    % Open quest dialog to inform operator that inializing failed    
    choice = questdlg('Arduino kunne ikke initialiseres. Kontroller forbindelse. Vil du proeve igen?','Fejl: Arduino kunne ikke initialiseres', 'Ja','Nej','Ja');

        switch choice
            % Do nothing - retry inializing again
            case 'Ja'
        
            % Returns to end function
            case 'Nej'
            return
        end
    end
end
\end{lstlisting} 

\newpage
\subsection{constants}
Denne funktion er en hjælpefunktion, som indeholder alle konstanter der anvendes rundt omkring i programmet, herunder strings og variabler. Dette gøres for, at mindske mængden af hard-kodede variabler ude i de enkelte Matlab funktioner og gøre programmet lettere at vedligeholde. Funktionen kaldes ved opstart af programmet i GUI'ens openingFcn callback funktion, som eksekveres inden GUI'en gøres synlig. I funktionen \textit{constants} gemmes de specificerede variabler i et struct kaldet \textit{constants} i handles (se \ref{software_datastruktur}). 

\newpage
\subsubsection{cameraFeed} \label{subsub:camfeed}
Denne funktion er implementeret udfra beskrivelsen i designdokumentet (\ref{subsub:Dcamerafeed} s. \pageref{subsub:Dcamerafeed}). Dog er den ændret i forhold til ikke at modtage et feed fra kameraet, men i stedet indlæses et billede fra en prædefineret mappe. 
Funktionen har handles som input og output. 

I et while loop itereres der gennem mappen, hvor en counter variabel definerer hvilket billede der skal indlæses. Efter dette nedskaleres billedet til halv størrelse med \textit{imresize}. Dette gøres for, at optimere tiden det tager at behandle billedet, samt vise det på GUI. I denne forbindelse kunne der være en potentiel risiko for, at information gik tabt på billedet i nedskalleringen. Dette blev testet på en række billeder for, at se om algoritmen (beskrevet i \ref{subsub:IMdetectIslets} ville fejle på det nedskallerede billede, samt om tiden for billedebehandlingen blev forbedret. \fxnote{Mangler dokumentation i form af billeder og tic toc} Billedet gives herefter som parameter til funktionen detectIslet (\ref{subsub:IMdetectIslets}). %I while loopet kaldes der til sidst pause(0,1), som pauser while loopet i 0,1 sekund. Dette simulerer at kameraet optager med 10 f/s. 

Matlab implementeringen er vedlagt i bilag \ref{bilag:cameraFeed}.

%\subsubsection{detectIslets}
\newpage
\subsubsection{detectIslets}
\label{subsub:IMdetectIslets}
I denne funktion sker selve billedesegmenteringen af de langerhanske øer. Selve segmenteringen sker via en række morfologiske operationer, som er nærmere beskrevet i dette afsnit. For at illustrere effekten af de enkelte steps er der i løbet af afsnittet vist billeder.

I figur \ref{fig:im} er det oprindelige billede vist. Billedet danner udgangspunktet for selve segmenteringen. Billedet indeholder én langerhansk ø. 

%\begin{lstlisting} 
% Converts the image to logical, based on threshold. 0.72 indicates pixels
% with luminance level above 0,72 (255*0.72 = 183) is converted to a 1. Pixels below this 
% level is converted to 0 
%bw = im2bw(im,0.72);
%\end{lstlisting} 

\begin{figure}[H]
	\centering
	\includegraphics[width=0.6\textwidth]{billeder/software/im.png}
	\caption{Oprindelige billede}
	\label{fig:im}
\end{figure}

Første step af billedesegmenteringen består i, at konvertere det oprindelige billede (\ref{fig:im}) til en logisk maske. Til dette er Matlab funktionen \textit{im2bw} anvendt. 
Effekten af im2bw er vist i figur \ref{fig:im2bw}. Funktionen konvertere alle pixels med en lysstyrke over det angivne threshold (0,72) med 1, og pixels under med 0. Input billedet er et 8 bit billede, dermed indikerer et threshold på 0,72, at pixel værdien skal være over 183 for at blive inkluderet i masken. I nedenstående kode er implementeringen vist.

\begin{lstlisting} 
% Converts the image to logical, based on threshold. 0.72 indicates pixels
% with luminance level above 0,72 (255*0.72 = 183) is converted to 1. Pixels below this 
% level is converted to 0 
bw = im2bw(im,0.72);
\end{lstlisting} 

\begin{figure}[H]
	\centering
	\includegraphics[width=0.6\textwidth]{billeder/software/im2bw.png}
	\caption{Billede konverteret til logisk}
	\label{fig:im2bw}
\end{figure}

I næste step bliver unødigt støj fra masken fjernet. Til dette anvendes funktionen \textit{bwareaopen}. Denne fjerner alle objekter i masken under 100 sammenhørende px. Effekten af denne funktion er vist i figur \ref{fig:bwarea}. Herudover kaldes funktionen bwlabel, som indeksere hvert af de sammenhørende objekter i masken. Dermed får hvert objekt i figur \ref{fig:bwarea} et unikt index fra 1 til 5. 
\begin{lstlisting} 
% Removes all objects below 100 connected pixels
bw = bwareaopen(bw,100);

% Label connected components
L = bwlabel(bw);
\end{lstlisting} 

\begin{figure}[H]
	\centering
	\includegraphics[width=0.6\textwidth]{billeder/software/bwarea.png}
	\caption{Små objekter fjernet}
	\label{fig:bwarea}
\end{figure}

I næste step sker selve ø detekteringen. Detektionen baseres på, at øerne har en mere cirkulær form end resten af vævet. Derfor hentes der egenskaber fra objekterne i masken vha. matlab funktionen \textit{regionsprops}. De egenskaber der hentes er med til, at beskrive hvor cirkulært objektet er. De egenskaber som hentes er arealet, center positionen, omkredsen og excentriciteten. Excentriciteten beskriver hvor langstrakt objektet er. Hvis excentriciteten er 0 er det en perfekt cirkel mens det vil være en langstrakt ellipse hvis værdien er 1.

Udfra arealet og omkredsen af objekterne anvendes de 2 nedenstående formler til bestemmelse af 2 værdier for radiusen af objektet:
\begin{align}
Areal = R1^2*\pi => R1 = \sqrt{\frac{\text{Areal}}{\pi}}
\end{align}
\begin{align}
Omkreds = 2*R2*\pi => R2 = \frac{\text{Omkreds}}{2*\pi}
\end{align}

Udfra disse 2 radius værdier, vil det objekt, hvor der er mindst forskel i mellem de 2, være det objekt der er mest cirkulært. Til at illustrere dette er der i figur \ref{fig:circleelip} vist en cirkel og en ellipse begge med samme areal. Udfra \textit{regionsprops} og de 2 formler for radius kan der bestemmes 2 radiuser for henholdsvis cirklen og ellipsen. Forskellen bestemmes ved, at trække de 2 beregnede værdier fra hinanden. I udregningerne nedenfor er den første kolonne værdier for cirklen, mens den anden er for ellipsen. Som det ses er forskellen mindst ved cirklen, hvilket indikerer, at den er mest cirkulær.
  
\begin{lstlisting} 
r1 =  100.0017   99.9380

r2 =  99.7978  105.7890

rDif = 0.2039    5.8510
\end{lstlisting} 

\begin{figure}[H]
	\centering
	\includegraphics[width=0.6\textwidth]{billeder/software/circleellipse.png}
	\caption{Cirkel sammenlignet med ellipse}
	\label{fig:circleelip}
\end{figure}

Indekset for det objekt med mindst forskel i radius og indekset for det objekt med den mindste excentricitet gemmes i variabler. I nedenstående kode er det vist, hvordan dette er implemeneteret i Matlab.

\begin{lstlisting} 
stats = regionprops(bw,'Area','Centroid','Perimeter','Eccentricity');

r1 = sqrt([stats.Area]/pi);
r2 = [stats.Perimeter]/(2*pi);
% The absolute difference between the 2 radius is calculated
rDif = abs(r1-r2);
% The index of the lowest difference is returned
[~, idx] = min(rDif);
% The index of the object with the lowest eccentricity is returned 
[~, idx2] = min([stats.Eccentricity]);
\end{lstlisting} 

I det sidste step i segmenteringen af den langerhanske ø kontrolleres det om radius forskellen og excentricitet ligger under nogle fast definerede grænseværdier. Værdierne er fundet ved, at analysere en række billeder og deres objekters radius og excentricitet. Radius og excentriciteten bliver gemt i logfilen, så de kan analyseres for senere at justere grænseværdierne. Hvis objektets værdier er under disse grænseværdier er en ø detekteret. Når en ø er detekteret illustreres det med en grøn ring på GUI. Til dette er funktionen \textit{viscircles} anvendt, hvor center positionen og radius for objektet anvendes. 

For at løse udfordringen med, at den samme ø vil optræde på det næste billede, er der implementeret et smalt detekteringsvindue (100 pixels bredt). Når øen er detekteret inden for dette område bliver variablen isletDetected sat til true. Denne variabel anvendes til styring af ventilen. Implementingen af dette er vist i nedenstående kode.


\begin{lstlisting} 
% If the difference between radius AND eccentricity is below the defined
 % values an islet has been detected
 if rDif(idx) < 0.45 && stats(idx2).Eccentricity <0.51
     
 % Show circle of cell on the 2 axes
 viscircles(h,stats(idx2).Centroid,r1(idx2)+10,...
 'LineStyle','-','edgecolor','g','LineWidth',1,'DrawBackgroundCircle',...
 false);
 viscircles(s,stats(idx2).Centroid,r1(idx2)+10,...
 'LineStyle','-','edgecolor','g','LineWidth',1,'DrawBackgroundCircle',...
 false);

     
      % Detection window  (0 to 100 pixels)
      if stats(idx2).Centroid(1) <=100 
      handles.flag1 = true;
      handles.count = handles.count+1; 
      set(handles.txtIslets,'String',num2str(handles.count));
      handles.isletDetected = true;
      end
\end{lstlisting} 

Figur \ref{fig:segmented} viser slutresultatet af segmenteringen, hvor masken kun indeholder den langerhanske ø fra det oprindelige billede. 


\begin{figure}[H]
	\centering
	\includegraphics[width=0.6\textwidth]{billeder/software/segmented.png}
	\caption{Slutresultat af segmentering}
	\label{fig:segmented}
\end{figure}

I figur \ref{fig:finalimage} er det oprindelige billede vist, med markering af den detekterede ø.


\begin{figure}[H]
	\centering
	\includegraphics[width=0.6\textwidth]{billeder/software/finalimage.png}
	\caption{Oprindelige billede med detekteret ø}
	\label{fig:finalimage}
\end{figure}

\newpage
\textbf{Test af billedeprocessering}

En udfordring med billedeprocesseringen er, at hastigheden ikke må være for langsom i forhold til frameraten for kameraet. Som beskrevet under funktionen cameraFeed (\ref{subsub:camfeed} indlæses et nyt billede hvert 0,1 sekund. Billedeprocesseringen skal dermed være hurtigere end dette for at følge med. Til test af dette er Matlab funktionen tic/toc anvendt, som måler hvor langt tid billedprocessingen tager om at eksekvere. I figur \ref{fig:dataprocess} viser grafen hvor langt tid det har taget at behandle de enkelte billeder. Den røde streg viser den gennemsnitlige tid for processeringen. For den nuværende implementering tager billedeprocessingen 0,0104 sekund pr. billede, hvilket er indenfor grænsen på 0,1 sekund. Tiden vil variere alt efter tilgængelig processorkraft og om der skal udføres andre opgaver, eksempelvis opdatere figurer på GUI. 

\begin{figure}[H]
	\centering
	\includegraphics[width=0.6\textwidth]{billeder/software/dataprocessing_2.png}
	\caption{Test af tidsforbrug for billedeprocessering}
	\label{fig:dataprocess}
\end{figure}
 
\newpage
\subsubsection{loadCell}
\label{subsub:softwareloadcell}
Funktionen til load cellen er implementeret efter beskrivelsen i design dokumentet. Dens funktion er, at konvertere det analoge input (V) til indholdet (ml) i celleopløsningsbeholderen. Dette er implementeres ved en lineær model:
\begin{align}
mL = a*input+b \text{, hvor a er hældningen og b er skæringen med y aksen}
\end{align}
Det analoge input ganges altså med en faktor a plus et offset b for at konvertere spænding til antal ml. Nedenstående tabel viser indgangsspændingen for forskellige mængder i beholderen. Udfra disse data er der lavet en lineær regression for at finde hældningen a og skæringen b.
\begin{center}
		\begin{longtable}{ | m{3cm} | m{3cm}| } 
			\hline
			\textbf{ml i beholder} &\textbf{Analog input} \\ 
			\hline
			 \SI{0}{\milli\litre} & \SI{1.9487}{\volt} \\ 
			\hline
			 \SI{25}{\milli\litre} & \SI{2.0440}{\volt} \\ 
			\hline
			\SI{50}{\milli\litre} & \SI{2.1320}{\volt} \\ 
			\hline
			\SI{75}{\milli\litre} & \SI{2.2297}{\volt} \\ 
			\hline
			\SI{100}{\milli\litre} & \SI{2.3109}{\volt} \\ 
			\hline
			\SI{125}{\milli\litre} & \SI{2.4071}{\volt} \\ 
			\hline
			\SI{150}{\milli\litre} & \SI{2.4961}{\volt} \\ 
			\hline
			\SI{175}{\milli\litre} & \SI{2.5821}{\volt} \\ 
			\hline
			\SI{200}{\milli\litre} & \SI{2.6760}{\volt} \\ 
			\hline
			\SI{225}{\milli\litre} & \SI{2.7654}{\volt} \\ 
			\hline
			\SI{250}{\milli\litre} & \SI{2.8587}{\volt} \\ 
			\hline
			\caption{Kalibreringsdata for loadcellen}
			 		\end{longtable}
\end{center}

I Matlab er funktionen \textit{fitlm} anvendt til at finde det bedste lineære fit. Regressionen er baseret på Least Square metoden \citep{least}.
Ud fra beregningerne i Matlab er hældningen a og skæringen b fundet til hhv:
\begin{align}
a = 276.14
\end{align}
\begin{align}
b = -539.02
\end{align}
Den endelige funktion er dermed givet ved:
\begin{align}
ml = 276.140*input-539.02
\end{align}

Figur \ref{fig:loadcellcalib} viser den lineære funktion, samt de enkelte data punkter fra tabellen. 
\begin{figure}[H]
	\centering
	\includegraphics[width=0.6\textwidth]{billeder/software/calibration-crop.pdf}
	\caption{Kalibrering af load cell}
	\label{fig:loadcellcalib}
\end{figure}

For at reducere støj og mindske følsomheden overfor hurtigere ændringer i indgangsspændingen er der implementeret en midling af de seneste 10 målinger. Dette er med til, at gøre konverteringen mere robust overfor støj.  

\newpage
\subsubsection{valveControl} \label{software_ventil}
I denne funktion styres åbning og lukning af ventilen. Til dette anvendes funktionen writeDigitalPin fra Arduino Support pakken, som bruges til at sætte pin'en høj (5V) og lav (0V). For at time hvornår ventilen skal lukke igen anvendes 2 timer objekter. En timer har den fordel, at den kan køre i baggrunden og dermed ikke blokkerer eksekvering af anden kode i programmet. Til et timer objekt kan der knyttes 3 callback funktioner, hhv.: \textbf{startFcn}, \textbf{TimerFcn} og \textbf{stopFcn}. Callback funktionerne specificerer hvad der skal ske ved forskellige tidspunkter. Timerens \textbf{startFcn} anvendes ikke ved de 2 timer objekter. I stedet defineres et \textbf{startDelay} som specificerer, hvor langt tid der skal gå før timerens egentlige callback funktion eksekveres. Det er dette delay der afgør, hvornår ventilen skal åbne eller lukke igen. Timerens \textbf{TimerFcn} sætter altså pin'en høj(5 V) for åbning og lav(0 V) for at lukke den igen.
Figur \ref{fig:timer} viser rækkefølgen for hvornår de enkelte timer funktioner eksekveres. 
\begin{figure}[H]
	\centering
	\includegraphics[width=0.6\textwidth]{billeder/software/timer-crop.pdf}
	\caption{Timer synkronisering}
	\label{fig:timer}
\end{figure}

Den første timers opgave er at åbne for ventilen. I denne timers \textbf{stopFcn}, startes den anden timer, som lukker ventilen igen. I ventilens design dokument, afsnit \ref{design_ventil}, blev det beregnet hvor langt tid der skulle gå fra en ø var detekteret til ventilen skulle åbne (54,475 ms). Ligeledes blev det beregnet, hvor længe ventilen skulle stå åben før dens volumen er tømt igen (144 ms). Til at beregning af timernes \textbf{startDelay} er nedenstående formler anvendt: 

\begin{align}
\text{Timer 1 delay}=54,475ms-20ms-26ms=8,475ms
\label{eq:timerdelay1}
\end{align} 
\begin{align}
\text{Timer 2 delay}=144ms-30ms-26ms= 88ms
\label{eq:timerdelay2}
\end{align} 

Fra ventilens datablad (bilag \ref{bilag:161T031}) ses det, at åbningstiden for ventilen er op til 20ms og en lukketid på op til 30ms. Disse tider trækkes fra i begge tidsforsinkelser. Herudover trækkes forsinkelsen over seriel forbindelsen mellem Matlab og Arduinoen. Denne tid er fundet via en unittest af ventilen, som vist i figur \ref{fig:serielconn}.   

I figur \ref{fig:timerdelay} er enhedstesten for ventilen vist. Her er timerne implementeret med ovenstående delays. Som det ses er tiden, hvor ventilen er åben målt til 112 ms. Dette resultat stemmer fint overens med Timer 2 delay på 88 ms (formel \ref{eq:timerdelay2}) plus forsinkelsen på 26 ms som det tager at sende en kommando fra Matlab til Arduinoen. Dette giver en total tid på 114 ms, som ligger tæt på de målte 112 ms. 

 \begin{figure}[H] \centering
\begin{minipage}[b]{0.48\textwidth} \centering
\includegraphics[width=1.00\textwidth]{billeder/timerdelay.png} % Left picture
\end{minipage} \hfill
\begin{minipage}[b]{0.48\textwidth} \centering
\includegraphics[width=1.00\textwidth]{billeder/serialconn.png} % Right picture
\end{minipage} \\ % Captions og labels
\begin{minipage}[t]{0.48\textwidth}
\caption{Tid hvor ventilen er åben. Timer 1 sætter pin'en høj (5V), for herefter at starte Timer 2. Når Timer 2's tid delay er gået sættes pin'en lav (0V). Den endelige tid ventilen er åben er 112 ms.} % Left caption and label
\label{fig:timerdelay}
\end{minipage} \hfill
\begin{minipage}[t]{0.48\textwidth}
\caption{Tidsforsinkelse over seriel forbindelsen mellem Matlab og Arduino. På figuren sendes skiftevis en kommando til at sætte pin'en høj (5V) og en kommando til at sætte pin'en lav (0V) til Arduinoen. Periodetiden er på 52 ms.} % Right caption and label
\label{fig:serielconn}
\end{minipage}
\end{figure}


Nedenstående kode viser, hvordan timerne er implementeret i Matlab.  
\begin{lstlisting} 
% Create timers - with startDelay (defined in handles.constants)
tHigh = timer('StartDelay',handles.constants.tHighDelay);
tLow = timer('StartDelay',handles.constants.tLowDelay);

% Assign timerfunction - to set pin HIGH (5V) to open valve and LOW (0V)
% to close valve
tHigh.TimerFcn = @(x,y) a.writeDigitalPin(handles.constants.valvePin,1);
tLow.TimerFcn = @(x,y) a.writeDigitalPin(handles.constants.valvePin,0);

% StopFcn - starts tLow when tHigh is done executing
tHigh.StopFcn = @(x,y)start(tLow);

% Start first timer
start(tHigh)

% isletDetected boolean is set to false
handles.isletDetected=false;
\end{lstlisting} 

Implementeringen er vedlagt i bilag \ref{bilag:matlab}.

\newpage
\subsubsection{pumpControl}
Funktionen til styring af pumpen anvendes funktionen \textit{writePWMDutyCycle} fra Arduino support pakken. Som input til denne funktion angives pin'en og PWM værdien, som en værdi mellem 0 og 1. Som standard er pumpen sat til en flow hastighed på 50 ml/min, hvilket er svarer til en PWM værdi på 0.9. 


\subsubsection{exportData}
Denne funktion gemmer data omkring sorteringscyklussen, som en .csv fil. Opbygningen af filen er nærmere beskrevet i kravspecifikationen (\ref{subsub:software}). Til at gemme filen anvendes Matlab funktionen \textit{writetable}. Da data om sorteringen er gemt, som et struct i handles konverteres structet til en tabel med funktionen \textit{struct2table}. Som filnavn anvendes starttiden for sorteringscyklussen. Denne værdi hentes fra handles.

Når filen er gemt informeres operatøren via en messageboks. 

\subsubsection{areaConverter}
Denne funktion bruges til at konvertere minimum og maksimum størrelserne defineret i systemets indstillingerne til et minimum og maksimum areal i pixels.  Indstillingsværdierne der er angivet i \SI{}{\micro\meter} skal konverteres til antal pixels. Da billederne er genereret er det ikke målt, hvor mange \SI{}{\micro\meter} der er pr. pixel, derfor er det implementeret, at 10 \SI{}{\micro\meter} svarer til 1 pixels på billedet. Nedenstående formel viser, hvordan konverteringen fra \SI{}{\micro\meter} til pixels er bestemt: 
\begin{align}
\text{Areal i pixels} = (\frac{\text{Diameter i \SI{}{\micro\meter}}}{\frac{2}{10}})^2*\pi
\end{align}
En konvertering med en diameter på 100 \SI{}{\micro\meter} giver dermed et areal på 78 pixels. Ud fra data om øerne i billederne har den mindste ø et areal på 100 pixels.  

\newpage
\subsection{GUI}
Dette afsnit indeholder en oversigt over hvordan de forskellige GUI vinduer er implementeret.
\subsubsection{Hovedvindue}
\begin{figure}[H]
	\centering
	\includegraphics[width=1\textwidth]{billeder/software/gui_main.png}
	\caption{Hovedvinduet på GUI}
	\label{fig:finishedGUI}
\end{figure}

\subsubsection{Indstillingsvindue}
\begin{figure}[H]
	\centering
	\includegraphics[width=0.4\textwidth]{billeder/software/settings.png}
	\caption{Indstillingsvindue på GUI}
	\label{fig:finishedSettings}
\end{figure}


\newpage
\subsection{Software enhedstest og integrationstest}
Dette afsnit beskriver hvordan softwaren er testet i implementeringsfasen. I gennem udviklingen af de enkelte funktioner er der løbende testet om funktionerne giver de forventede outputs, som beskrevet i designdokumentet. Til dette er der anvendt breakpoints. Matlabs \textit{tic/toc} funktion er også anvendt til at sikre eksekveringen af tidskritiske funktioner ikke tager for langt tid. 

Til at teste softwarens interface med Arduinoen er LED'er brugt til at koble på de pinne hvor ventil, pumpe og kameralys skal monteres. Dette gav mulighed for visuelt at se om pinnene på Arduinoen blev sat høj (5V) og lav(0V) som forventet. Herudover blev et potentiometer brugt som erstatning til vægtcellen. Dermed kunne indgangsspændingen manuelt justeres og det kunne dermed visuelt observeres at beholderens indhold blev ændret.

Til sidst i integrationstesten blev de egentlige hardware komponenter koblet på Arduinoens pinne og det blev testet om softwaren fortsat kørte efter hensigten.
%\newpage
%\subsection{Matlab callbacks}
%Dette afsnit beskriver callback funktionerne..
%%\subsubsection{•}
%\subsubsection{Start callback}
%
%\subsubsection{Stop callback}
%
%\subsubsection{Settings callback}