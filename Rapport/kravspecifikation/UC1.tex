\subsection{Use Case 1}
Dette er Use Case 1


\begin{table}[]
\centering
\label{my-label}
\begin{tabular}{|l|l|}
\hline
\textbf{Use Case 1  }           & \textbf{Påfyldning af celler} \\ \hline
Mål:                            &         \\ \hline
Initiering:                     &                            \\ \hline
Aktør:                          &                            \\ \hline
Startbetingelser:               &                            \\ \hline
Slutbetingelser ved succes:     &                            \\ \hline
Slutbetingelser ved undtagelse: &                            \\ \hline
Normalforløb:                   &                            \\ \hline
Undtagelser:                    &                            \\ \hline
\end{tabular}
\caption{Use Case 1}
\label{tab:uc1}
\end{table}

\begin{table}[]
\centering

\begin{tabular}{|l|l|}
\hline
\textbf{Use Case 1  }           & \textbf{Påfyldning af celler }\\ \hline
Mål:                            &               \\ \hline
Initiering:                     & test	\\ \hline
Aktør:                          &   Operatør                       \\ \hline
Startbetingelser:               &    Opløsningsbeholderen er tom        \\ \hline
Slutbetingelser ved succes:     &     Opløsningsbeholderen er fyldt      \\ \hline
Slutbetingelser ved undtagelse: &                            \\ \hline
Normalforløb:                   & Punktopstilling
\\ \hline
Undtagelser:                    &                            \\ \hline
\end{tabular}

\caption{My caption}
\label{my-label}
\end{table}

\begin{center}
		\begin{longtable}{ | m{4cm} | m{8cm}| } 
			\hline
			Use Case 1 & Påfyldning af celler \\
			\hline
			Mål & Påfyldning af celler  \\ 
			\hline
			Initiation &  Medicinsk personale\\
			\hline
			Actors and stakeholders & 
			\begin{itemize}
				\item Medicinsk personale(primær)
				\item Patient (sekundær)
			\end{itemize} \\ 
			\hline
			References & Use case 3 \\ 
			\hline
			Number of concurrent occurrences & En til mange\\ 
			\hline	
			Precondition & 
			\begin{itemize}
				\item Mode switch er sat til “\textit{Konditionering}”
			\end{itemize} \\ 
			\hline
			Postcondition & 
			\begin{itemize}
				\item 5 hele cyklus er gennemført og gemt på hukommelsen
			\end{itemize} \\ 
			\hline
			Main scenario & \begin{enumerate}
				\setlength\itemsep{0cm} % Decrease line distance
				\item \textit{Medicinsk personale} placerer manchetten på patienten
				\item Knappen [Start/Stop] trykkes
				\item Et nyt patient ID genereres
				\subitem[Extension \#1] 
				\item Patient ID’et vises på skærmen
				\item Blodtrykket måles via \textit{use case 3}
				\subitem[Extension \#2]
				\item Blodtrykket vises på displayet og værdien gemmes i hukommelse
				\item Cuffen fyldes med luft til et tryk på 25 mmHG over systolisk tryk (minimum 180 mmHg)
			\end{enumerate} \\ 
			\hline
			Main scenario & \begin{enumerate}
				\setlength\itemsep{0cm} % Decrease line distance
				\setcounter{enumi}{7}				
				\item Tidsstempel gemmes når trykket er opnået
				\item Trykket opretholdes i 5 minutter(Okklusion) og resterende tid vises på displayet
				\item Blodtrykket måles via \textit{use case 3}
				fra punkt 2.
				\item Deflaterer cuffen helt og forbliver i dette stadie i 5 min(Reperfusion) Ved deflation start gemmes tidsstempel. Tid til næste okklusion vises på displayet
				\item Gentag punkt 7-11 (en cyklus) fire gange. Det nuværende cyklus nummer vises i displayet
			\end{enumerate} \\ 
			\hline
			Extensions & [Extension \#1] Et patient ID eksisterer allerede på apparatet. Der genereres ikke noget nyt patient ID.
			
			[Extension \#2] Blodtrykket kunne ikke måles. Gentag use case 3 hvis extension 2 ikke lige er eksekveret. Ellers skrives i display “FEJL kunne ikke måle blodtryk” og use casen stopper.  \\
			\hline
		\end{longtable}
		
	\end{center}