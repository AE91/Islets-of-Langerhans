\chapter{Kravspecifikation}

\section{Indledning}
Dette dokument indeholder kravspecifikationen for The Cell Collector(omtales herefter som systemet). Dokumentet er udarbejdet i samarbejde med kunden(Søren Gregersen) og specificerer kundens kvalitetskrav, samt funktionelle og ikke funktionelle krav. Der er sammen med kunden udarbejdet en accepttest, som har til formål at teste de specificerede krav i kravspecifikation.
\section{Versionshistorik}
Tabel...

\section{Systembeskrivelse}
Formålet med projektet er at udvikle et system til isolation af insulin producerende celler (Langerhanske Øer). Mange farmaceutiske virksomheder og forskningsafdelinger udfører forsøg på disse øer fra bl.a. rotter. Processen med isolering af Langerhanske øer startes ved operativt at fjerne pancreas, hvorefter vævet opløses vha. enzymet kollagenase. Når vævet er opløst fortyndes det yderligere inden det hældes i petriskåle. Øerne bliver herefter manuelt isoleret vha. mikroskop og diverse præcisions redskaber. Denne proces er både besværlig og tidskrævende. Formålet med projektet er derfor, at udvikle en ny metode til isolation af cellerne. Systemet skal indeholder en beholder til opløsningen med langerhanske øer. Denne opløsning skal føres ud gennem en tynd slange(<0,5mm)  forbi et kamera, hvor der ved hjælp af Matlab skal udføres billedprocessering. Billedebehandlingen skal genkende, hvornår der er en langerhanske ø. Derefter skal systemet frasortere denne, ved et ventil system der åbner på det rigtige tidspunkt. Til at skabe flowet i slangerne anvendes en pumpe.  Et krav til pumpen er at den skal være nænsom ved celleopløsningen, da de langerhanske øer er meget skrøbelige.
En automatiseret løsning af sorteringsprocessen kan bidrage med reducering af omkostningerne, give en mere ensartet sortering samt sikre dokumentation af de sorterede øer. Systemet kan fra et kommercielt synspunkt bidrage til basal forskning og til screening af nye medicinske præparater.

Indsæt figur!!
\section{Funktionelle krav}
\subsection{Aktør beskrivelse}
Systemets primære aktør er operatøren, som står for påfyldning af celler, start og stop af sorteringsprocessen. Operatøren har mulighed for at interagere med systemet via en grafisk brugergrænseflade. Systemets sekundære aktør er PC’ens filsystem, hvor der løbende gemmes en log over sorteringsprocessen.
\subsection{Use Case Diagram}
Indsæt figur!!
%\subsection{Use Case 1 - Start sorteringscyklus}
\label{uc:1}
\begin{center}
		\begin{longtable}{ | m{4cm} | m{8cm}| } 
			\hline
			Mål & Start sorteringscyklus \\ 
			\hline
			Initiering &  Use casen initieres af operatøren\\
			\hline
			Aktør & 
			Primær: Operatør
			
			 Sekundær: Kamera			  \\ 
			\hline
			Startbetingelser & The Cell Collector programmet er startet på computeren \\ 
			\hline	
			Slutbetingelser ved succes & Systemet starter med sorteringen af Langerhanske øer \\
			\hline
			Slutbetingelser ved undtagelse & N/A \\
			\hline
			Normalforløb & \begin{enumerate}
				%\setlength\itemsep{0cm} % Decrease line distance
				\item Operatør fylder celleopløsningsbeholderen
				\item Celleopløsningsbeholderen er fyldt
				\item Operatør starter sorteringscyklus ved at klikke på [Start]
				\subitem [Undtagelse 1: Wastebeholder er fyldt] 
				\item Systemet initialiserer Arduinoen
				\subitem [Undtagelse 2: Ingen forbindelse til Arduino]
				\item Systemet kontrollerer celleopløsningsbeholderen ved at konvertere spændingen (\SI{}{\volt})  til \SI{}{\milli\litre}, og vise beholderens indhold (\SI{}{\milli\litre}) på GUI
				\item Systemet initialiserer kameraet
				\subitem [Undtagelse 3: Kameraet initialiserer ikke]
				\item Systemet tænder for kamera lyset
				\item Systemet tænder for pumpen
				
			\end{enumerate} \\ 
			\hline
			Undtagelser & [Undtagelse 1: Wastebeholder er fyldt] 
			
			\begin{enumerate}
			\item Systembesked: Tøm venligst Wastebeholder før start
			\item Operatøren trykker “OK”
			\item Systemet fortsætter opstartprocessen
			\end{enumerate} 
			
			[Undtagelse 2: Ingen forbindelse til Arduino]
			
			\begin{enumerate}
			\item 1.	Systembesked: Ingen forbindelse til Arduino, kontrollér forbindelser.
			\end{enumerate} 
	
			[Undtagelse 3: Kameraet initialiseres ikke]
			
			\begin{enumerate}
			\item System fejlmeddelse: Kameraet er ikke initialiseret:
			\item Genstart initialisering af Kameraet
			\end{enumerate} \\
			\hline
		\end{longtable}
		
	\end{center}
	\pagebreak
\subsection{Use case 2 - Sortering af Langerhanske Øer}
\label{uc:2}
\begin{center}
		\begin{longtable}{ | m{4cm} | m{8cm}| } 
			\hline
			Mål & Sortere Langerhanske Øer \\ 
			\hline
			Initiering &  Use casen initieres af [UC 1: Startsorteringscyklus]\\
			\hline
			Aktør & Kamera \\ 
			\hline
			Startbetingelser & Systemet er startet og sorteringscyklussen er i gang\\ 
			\hline	
			Slutbetingelser ved succes & Systemet har isoleret en Langerhansk ø og ventilen er lukket \\
			\hline
			Slutbetingelser ved undtagelse & \\
			\hline
			Normalforløb & \begin{enumerate}
				%\setlength\itemsep{0cm} % Decrease line distance
				\item Systemet detekterer en Langerhansk ø
				\item Arduino sender signal til ventilen om åbning
				\item Ventilen åbner
				\item Arduino sender signal til ventilen om lukning
				\item Ventilen lukker
			\end{enumerate} \\ 
			\hline
			Undtagelser & \\
			\hline
		\end{longtable}
		
	\end{center}
	\pagebreak
\subsection{Use Case 3 - Stop sorteringscyklus}
\begin{center}
		\begin{longtable}{ | m{4cm} | m{8cm}| } 
			\hline
			Mål & Stop sorteringscyklus \\ 
			\hline
			Initiering &  Use casen initieres af operatøren\\
			\hline
			Aktør & Operatør \\ 
			\hline
			Startbetingelser & [UC 2: Sortering af Langerhanske Øer] er startet\\ 
			\hline	
			Slutbetingelser ved succes & [UC 2: Sortering af Langerhanske Øer] er stoppet \\
			\hline
			Slutbetingelser ved undtagelse & N/A \\
			\hline
			Normalforløb & \begin{enumerate}
				\setlength\itemsep{0cm} % Decrease line distance
				\item Operatør stopper sorteringscyklussen ved at trykke på [Stop]
				\subitem [Undtagelse 1: Tom celleopløsningsbeholder]
				\item Systemet slukker for pumpen
				\item Systemet slukker for kameraet
				\item Systemet slukker for kamera lyset
				\item Systemet slukker for Arduino
			\end{enumerate} \\ 
			\hline
			Undtagelser & TODO!!\\
			\hline
		\end{longtable}
		
	\end{center}
	\pagebreak
\subsection{Use case 4 - Indstillinger}
\begin{center}
		\begin{longtable}{ | m{4cm} | m{8cm}| } 
			\hline
			Mål & Ændre systemets indstillinger \\ 
			\hline
			Initiering &  Use casen initieres af operatør\\
			\hline
			Aktør & Operatør \\ 
			\hline
			Startbetingelser & [UC 2: Sortering af Langerhanske Øer] er endnu ikke startet\\ 
			\hline	
			Slutbetingelser ved succes & Systemets indstillinger er ændret \\
			\hline
			Slutbetingelser ved undtagelse & Systemets indstillinger er uændret \\
			\hline
			Normalforløb & \begin{enumerate}
				%\setlength\itemsep{0cm} % Decrease line distance
				\item Operatøren klikker på [Indstillinger]
				\item Et nyt vindue åbner med systemets indstillinger.
				\item Operatøren vælger de ønskede indstillinger, og trykker [Gem indstillinger]
				\subitem [Undtagelse 1: Operatøren klikker [Annuller]]
				\item Systemets indstillinger gemmes.
			\end{enumerate} \\ 
			\hline
			Undtagelser & [Undtagelse 1: Operatøren klikker “Annuller”]
			
			\begin{enumerate}
			\item Systemet lukker Indstillingsvinduet og indstillingerne er uændret.
			\end{enumerate} \\
			\hline
		\end{longtable}
		
	\end{center}
	\pagebreak


\section{Ikke funktionelle krav}
Todo: kvalitetskrav fra kunde
\subsection{Hardware}

\subsubsection{Microcontroller} \label{subsub:mcu}
\begin{enumerate}
\item Atmega328p (Arduino)
\end{enumerate}

\subsubsection{Pumpe} \label{subsub:pump}
\begin{enumerate}
\item Pumpe flow: <50ml / min 
\item Størrelse på studserne skal kunne tilpasses slangerne 
\end{enumerate}

\subsubsection{Slanger} \label{subsub:tubes}
\begin{enumerate}
\item Slangerne skal have en indre diameter > 300µm
\item Kameraet skal kunne detektere langerhanske øer igennem slangen, evt. vha. glasrør
\end{enumerate}

\subsubsection{Beholdere} \label{subsub:container}
\begin{enumerate}
\item Celleopløsningsbeholder skal have størrelse  > 250 mL
\item Wastebeholder skal have en størrelse dobbelt så stor som celleopløsningsbeholderen: > 500 mL
\end{enumerate}

\subsubsection{Ventil} \label{subsub:valve}
\begin{enumerate}
\item 3-vejs, dvs. 1 tilgang og kobling mellem 2 udgange
\item Studserne skal kunne tilpasses slangerne
\item Skal være til væske
\item Lukke og åbne tid skal være >50ms
\end{enumerate}

\subsubsection{Kamera} \label{subsub:camera}
\begin{enumerate}
\item Kameraet til kunne detektere langerhandske øer mellem 100 og 300um
\item Kameraet skal have en zoom funktion, som et mikroskop
\item Kameraet skal have et USB interface
\end{enumerate}


\subsection{Software} 

\subsubsection{Dataformat og struktur} \label{subsub:software}
\begin{enumerate}
\item CSV format med kommasepareret delimiter. 
\item Filnavn: Dato og starttidspunkt for sorteringscyklus.
\item Header indeholdende opsætningsindstillinger.
\item Filen er opbygget med følgende kolonner: 
\begin{enumerate}
\item Tidsstempel i formatet DD-MM-YYYY-hh:mm:ss
\item Ø størrelse
\item Radius
\item Omkreds
\item Middelintensitet
\end{enumerate}
\end{enumerate}



\section{Projektafgrænsning}
Til at afgrænse kravene i projektet er der anvendt MoSCoW metoden. Denne metode er brugt for at give en struktureret oversigt over hvilke krav der er vigtigst at få opfyldt inden for tidsrammen og hvilke der evt. kan implementeres senere hvis tiden er til det.

INDSÆT FIGUR!!
\section{Samarbejdspartner}
Gruppens kunde er Søren Gregersen, overlæge på Medicinsk Endokrinologisk Afdeling, Aarhus Universitetshospital. Det er i samarbejde med ham at projektet er blevet specificeret, samt hvilke krav der er til den endelige prototype.
Samuel Alberg Thrysøe er gruppens projektvejleder. Der afholdes ugentligt et vejledermøde, hvor gruppen giver status på projektet og hvor der diskuteres forskellige problemstillinger. 
Simon Vammen Grønbæk og Karl Johan Schmidt fungerer som projektets review gruppe. Der holdes møde hver anden uge omhandlende aftalt dagsorden. Formålet med review gruppen er at få konstruktiv feedback på evt. rettelser, opbygning af rapport og generel forståelse.
