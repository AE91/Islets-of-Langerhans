\section*{Resume}
\textbf{Baggrund}
Insulin er et essensielt hormon til regulering af glukose i blodet. Et fald i insulin produktionen ved nedsat funktion i de langerhanske øer kan føre til den livstruende sygdom \textit{diabetes mellitus}. Til at undersøge sygdommen og funktionen af de insulin producerende øer laves der videnskabelige forsøg med langerhanske øer fra mus. Denne proces foregår ved operativt at fjerne pankreas for herefter at opløse den ved hjælp af enzymet \textit{collagenase}. De enkelte øer isoleres herefter ved manuelt, at plukke dem fra en petriskål. Denne proces er både besværlig og tidskrævende. Derfor ønskes der nye metoder til isolering af langerhanske øer.

\textbf{Metoder} I gennem en agil udviklingsproces er der udviklet en \textit{Proof of Concept} prototype til automatisk isolering af langerhanske øer. Udviklingsfasen bestod af overordnede af fem faser, hhv. koncept fase, kravspecifikation, design, implementering og accepttest. Projektets overordnede tidsplan er opbygget som en stage gate model, hvor de enkelte udviklingsfaser er fastlagt med deadlines. I samarbejde med projektets review gruppe er de enkelte faser reviewed efter endt deadline. For at holde styr på arbejdsopgaverne er scrum værktøjet PivotalTracker anvendt, som har givet et overblik over de enkelte ugers sprints.

\textbf{Resultater og diskussion} Den udviklede prototype består af et software program udviklet i \textit{Matlab}, samt en række hardware komponenter. Prototypen virker vha. af billedprocessering, som detekterer de enkelte øer når de passerer et kamera. Ved detektion isoleres øen vha. en ventil i en seperat beholder. En enhedstest af det indkøbte kamera viste, at det ikke var tilstrækkelig kvalitet til detektering af øerne. Derfor er der i prototypen implementeret en funktion til simulering af kameraet ud fra genererede billeder.
  
Herudover er en cost-benefit analyse udarbejdet til at belyse, hvilke økonomiske fordele en automatiseret løsning vil have.   
%Projektet har yderligere givet et indblik i hvilke problemstillinger der skal løses inden en fungerende prototype kan implementeres i praksis, herunder hvorvidt øerne tager skade af processen. 


\textbf{Konklusion} Projektet har vist, at det er muligt at detektere langerhanske øer vha. billedprocessering ud fra genererede billeder. Systemet har desuden vist sig, at være velegnet til isolering af objekterne i simuleringsvæsken. Projektet har yderligere givet et indblik i hvilke problemstillinger der skal løses inden en fungerende prototype kan implementeres i praksis, herunder hvorvidt øerne tager skade af processen. 

Resultatet af cost-benefit analysen giver incitament til, at arbejde videre med en automatiseret løsning til isolering af øerne. 

