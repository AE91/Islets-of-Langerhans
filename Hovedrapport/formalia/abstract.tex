\section*{Abstract}
\textbf{Background}
Insulin is a hormone essential for regulating glucose levels in the blood. A decrease in insulin production in the pancreatic Langerhans islets can lead to the life-threatening disease \textit{diabetes mellitus}. To further investigate the disease researchers conduct experiments with islets isolated from mice. This is done by surgically removing the pancreas, dissolving the pancreatic tissue with the enzyme \textit{collagenase} and manually picking the islets from a suspension. This procedure is both cumbersome and time-intensive. Hence, new ways are warranted.   

\textbf{Methods}
Through an agile development process a \textit{Proof of Concept} prototype has been developed for automatic isolation of the islets of Langerhans. The development process consisted of five general phases: conceptual phase,  specification phase, system design phase, implementation phase and finally a system verification phase. The project was managed with a stage gate model where each development phase was given a specific deadline. In cooperation with a review group each phase was reviewed after the deadline. To provide an overview of the different tasks and sprints in the project the scrum based PivotalTracker was used.   


\textbf{Results \& discussion} The developed prototype consists of a software program developed in \textit{Matlab} and different Hardware components.  The system works by processing images taken from a microscope camera. If an islet is detected a mechanical valve will isolate the islet from the remaining fluid. However a unit test showed that the camera wasn't sufficient to detect the islets. In the final prototype a function has been implemented to simulate the camera by feeding a sequence of pre generated images.

In addition a cost-benefit analysis has disclosed the economic benefits of an automatic isolation method.

\textbf{Conclusion}
The project has shown that it is possible to detect islets of Langerhans through image processing of the generated images. However, the system has failed to isolate objects from the simulation fluid due to an obstruction in the mechanical valve. The project also showed which issues need to be addressed before the system can be implemented in practice. The results of the cost-benefit analysis has given incentive to do further research towards an automatic system for isolation of pancreatic islets. 