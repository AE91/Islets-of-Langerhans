\chapter*{Forord}

Denne rapport er udarbejdet som en del af et syvende semesters bachelorprojekt på Ingeniørhøjskolen, Aarhus Universitet. Rapporten er udarbejdet af en projektgruppe bestående af 2 sundhedsteknologistuderende. Projektet er udført i samarbejde med Søren Gregersen, overlæge på Medicinsk Endokrinologisk Afdeling på Aarhus Universitetshospital med hjælp fra Per. B. Jeppesen, Lektor ved Institut for Klinisk Medicin, Aarhus Universitet. Bachelorprojektet er udført i perioden 28. august 2015 til 16. december 2015, hvor forprojektet er udarbejdet i perioden 26. april 2015 til 15. juni 2015.  

Projektgruppen retter en stor tak til Søren Gregersen for samarbejdet, ligeledes skal der gives en tak til Per B. Jeppesen. Ydermere skal der lyde en varm tak til gruppens vejleder Samuel Thrysøe, der har hjulpet og støttet gruppen igennem hele processen. Endelig skal der gives en tak til reviewgruppen bestående af Simon Vammen Grønbæk og Karl-John Schmidt, som har givet konstruktiv kritik og rettelser. 


%Dette dokument indeholder projektdokumentationen for projektet \textit{Cell sorter for isolation of insulin producing cells}. Dokumentet indeholder kravspecifikation og accepttest for systemet, samt beskrivelse af projektets design og implementeringsfase. 

%Kravspecifikationen er udarbejdet i samarbejde med Søren Gregersen, overlæge på Medicinsk Endokrinologisk Afdeling, Aarhus Universitetshospital, der agerer som projektets kunde. 



\phantom{Luft}

\phantom{Luft}

\begin{table}[H]
	\centering
		\begin{tabular}{c c}
			\underline{\phantom{mmmmmmmmmmmmmm}} & \underline{\phantom{mmmmmmmmmmmmmm}}  \\
			Anders Toft Andersen			& Anders Esager		 			\\ 										\end{tabular}
\end{table}

\section*{Læsevejledning}
Rapporten indeholder primært metoder, resultater og diskussioner til produktet gruppen har udarbejdet. Der vil igennem rapporten fremtræde kildehenvisninger, som vil være samlet i en kildeliste bagerst i rapporten. Der er i rapporten anvendt kildehenvisning efter Harvardmetoden, i teksten refereres en kilde med [Efternavn, År]. Denne henvisning fører til kildelisten, hvor bøger er angivet med forfatter, titel, udgave og forlag, mens internetsider er angivet med forfatter, titel og dato. Til sidst i rapporten er bilagsliste, som anskueliggør filnavnene i den afleverede bilagsmappen. Diagrammerne udarbejdet i projektet er skrevet på engelsk. 

I bilagslisten forefindes alle filerne, der er afleveret ved siden af rapporten, herunder datablade, Matlab-kode, Eagle kilde-filer og Gerber-filer. Herudover er den udfyldte accepttest og fejlrapport vedlagt som bilag.

Ved siden af rapporten er vedlagt en video, som viser den udviklede prototype.

%Der vil igennem rapporten fremtræde kildehenvisninger, og disse vil være samlet i en kildeliste bagerst i rapporten. Der er i rapporten anvendt kildehenvisning efter Harvardmetoden, så i teksten refereres en kilde med [Efternavn, År]. Denne henvisning fører til kildelisten, hvor bøger er angivet med forfatter, titel, udgave og forlag, mens Internetsider er angivet med forfatter, titel og dato. Figurer og tabeller er nummereret i henhold til kapitel, dvs. den første figur i kapitel 7 har nummer 7.1, den anden, nummer 7.2 osv. Forklarende tekst til figurer og tabeller findes under de givne figurer og tabeller.
