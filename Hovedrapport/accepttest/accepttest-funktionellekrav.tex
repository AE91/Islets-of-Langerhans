\section{Accepttest af funktionelle krav}
Accepttesten udføres i samarbejde med vejlederen til projektet, som skal agerer kunde og dermed godkende testen. Testen udføres ved at følge handlingen i nedenstående tabeller, hvor det sammenlignes med det forventet resultat med det givende resultat i testen. Er resultatet, som forventet godkendes testen ved, at sætte flueben, dato og initialer for personen der udfører testen. Modsat hvis resultatet ikke er som forventet, skal testen markeres med rødt og der laves en fejlrapport. I fejlrapporten beskrives hvad der fejlede ved testen og en handlingsplan for hvad der skal ændres, for at få den godkendt. 
\subsection{Use case 1: Start sorteringscyklus}

	\begin{center}
		\begin{longtable}{ | m{4cm}| m{8.5cm}|} 
			\hline
			\textbf{Krav nr.} & 1.1 \& 1.2    \\ 
			\hline
			\textbf{Handling} &  Operatør fylder celle-opløsnings-beholderen   \\
			\hline
			\textbf{Forventet resultat} &  Celle opløsningsbeholderen er fyldt  \\
			\hline
			\textbf{Testmetode}  &  Celle opløsningsbeholderen fyldes med væske  \\
			\hline
			\textbf{Resultat}  &    \\
			\hline
			\textbf{Angiv godkendelse} &     \\
			\hline
			\textbf{Initialer} &     \\
			\hline
			\textbf{Dato} &    \\
			\hline
		\end{longtable}
	\end{center}
	
	
	\begin{center}
		\begin{longtable}{ | m{4cm}| m{8.5cm}|} 
			\hline
			\textbf{Krav nr.} & 1.3    \\ 
			\hline
			\textbf{Handling} &  Operatør starter sorteringscyklus ved at klikke på [Start]  \\
			\hline
			\textbf{Forventet resultat} &  Opstarts processen i gang sættes.  \\
			\hline
			\textbf{Testmetode}  & Knappen [Start] trykkes, resultatet observeres ved tekstboks på GUI, med teksten \textit{systemet starter op}.   \\
			\hline
			\textbf{Resultat}  &    \\
			\hline
			\textbf{Angiv godkendelse} &     \\
			\hline
			\textbf{Initialer} &     \\
			\hline
			\textbf{Dato} &    \\
			\hline
		\end{longtable}
	\end{center}
	
	\newpage
	
	\begin{center}
		\begin{longtable}{ | m{4cm}| m{8.5cm}|} 
			\hline
			\textbf{Krav nr.} & 1.4    \\ 
			\hline
			\textbf{Handling} &  Systemet initialisere Arduinoen   \\
			\hline
			\textbf{Forventet resultat} &  Arduino initialiseret signal modtages og gives til GUI  \\
			\hline
			\textbf{Testmetode}  & Det observeres på GUI at Arduinoen er initialiseret, i en tekstboks med testen \textit{Arduino er initialiseret}.   \\
			\hline
			\textbf{Resultat}  &    \\
			\hline
			\textbf{Angiv godkendelse} &     \\
			\hline
			\textbf{Initialer} &     \\
			\hline
			\textbf{Dato} &    \\
			\hline
		\end{longtable}
	\end{center}
	
	\begin{center}
		\begin{longtable}{ | m{4cm}| m{8.5cm}|} 
			\hline
			\textbf{Krav nr.} & 1.5    \\ 
			\hline
			\textbf{Handling} &  Systemet kontrollerer celle-opløsnings-beholderen  \\
			\hline
			\textbf{Forventet resultat} &  Antal ml i celleopløsningsbeholderen vises på GUI.  \\
			\hline
			\textbf{Testmetode}  & Der hældes 100 ml i celleopløsningsbeholderen, det observeres på GUI om der vises 100 ml $\pm$ 10 ml   \\
			\hline			
			\textbf{Resultat}  &    \\
			\hline
			\textbf{Angiv godkendelse} &     \\
			\hline
			\textbf{Initialer} &     \\
			\hline
			\textbf{Dato} &    \\
			\hline
		\end{longtable}
	\end{center}	
	
	\begin{center}
		\begin{longtable}{ | m{4cm}| m{8.5cm}|} 
			\hline
			\textbf{Krav nr.} & 1.6    \\ 
			\hline
			\textbf{Handling} &  Systemet initialiserer kameraet  \\
			\hline
			\textbf{Forventet resultat} &  Kamera Initialiseret signal modtages og gives til UI  \\
			\hline
			\textbf{Testmetode}  & Det observeres på GUI’en at kameraet er initialiseret, i en tekstboks med testen \textit{Kameraet er initialiseret}   \\
			\hline
			\textbf{Resultat}  &    \\
			\hline
			\textbf{Angiv godkendelse} &     \\
			\hline
			\textbf{Initialer} &     \\
			\hline
			\textbf{Dato} &    \\
			\hline
		\end{longtable}
	\end{center}

\newpage

	\begin{center}
		\begin{longtable}{ | m{4cm}| m{8.5cm}|} 
			\hline
			\textbf{Krav nr.} & 1.7    \\ 
			\hline
			\textbf{Handling} &  Systemet tænder kamera lyset \\
			\hline
			\textbf{Forventet resultat} &  Lyset tænder  \\
			\hline
			\textbf{Testmetode}  & Kamera lyset observeres   \\
			\hline
			\textbf{Resultat}  &    \\
			\hline
			\textbf{Angiv godkendelse} &     \\
			\hline
			\textbf{Initialer} &     \\
			\hline
			\textbf{Dato} &    \\
			\hline
		\end{longtable}
	\end{center}

	\begin{center}
		\begin{longtable}{ | m{4cm}| m{8.5cm}|} 
			\hline
			\textbf{Krav nr.} & 1.8    \\ 
			\hline
			\textbf{Handling} &  Systemet tænder pumpen \\
			\hline
			\textbf{Forventet resultat} &  Pumpen starter \\
			\hline
			\textbf{Testmetode}  & Det observeres ved at se flowet i slangerne stiger  \\
			\hline
			\textbf{Resultat}  &    \\
			\hline
			\textbf{Angiv godkendelse} &     \\
			\hline
			\textbf{Initialer} &     \\
			\hline
			\textbf{Dato} &    \\
			\hline
		\end{longtable}
	\end{center}
	
		
	\begin{center}
		\begin{longtable}{ | m{4cm}| m{8.5cm}|} 
			\hline
			\textbf{Krav nr.} & Undtagelse 1.1   \\ 
			\hline
			\textbf{Handling} & Waste-beholderen er fyldt  \\
			\hline
			\textbf{Forventet resultat} & System-besked: Tøm waste-beholderen  \\
			\hline
			\textbf{Testmetode}  & Der gennemføres 2 sorteringscyklusser. Ved start af 3. sorteringscyklus observeres resultatet på GUI ved en messagebox.   \\
			\hline
			\textbf{Resultat}  &    \\
			\hline
			\textbf{Angiv godkendelse} &     \\
			\hline
			\textbf{Initialer} &     \\
			\hline
			\textbf{Dato} &    \\
			\hline
		\end{longtable}
	\end{center}
	
\newpage	
	
	\begin{center}
		\begin{longtable}{ | m{4cm}| m{8.5cm}|} 
			\hline
			\textbf{Krav nr.} & Undtagelse 1.2 \& 1.3   \\ 
			\hline
			\textbf{Handling} & Operatør trykker [OK]  \\
			\hline
			\textbf{Forventet resultat} & Systemet fortsætter opstartprocessen, observeres ved at næste komponent initialiseres, skrives i tekstboks på GUI \\
			\hline
			\textbf{Testmetode}  & Knappen [OK] trykkes på messagebox.   \\
			\hline
			\textbf{Resultat}  &    \\
			\hline
			\textbf{Angiv godkendelse} &     \\
			\hline
			\textbf{Initialer} &     \\
			\hline
			\textbf{Dato} &    \\
			\hline
		\end{longtable}
	\end{center}	

	\begin{center}
		\begin{longtable}{ | m{4cm}| m{8.5cm}|} 
			\hline
			\textbf{Krav nr.} & Undtagelse 2   \\ 
			\hline
			\textbf{Handling} & Forbindelsen til Arduino frakobles  \\
			\hline
			\textbf{Forventet resultat} & System besked: Arduino kunne ikke initialiseres. Kontrollér forbindelse. Vil du prøve igen? \\
			\hline
			\textbf{Testmetode}  & USB kablet til Arduinoen frakobles før der trykkes [Start]. Resultatet observeres på GUI ved en dialogboks.   \\
			\hline
			\textbf{Resultat}  &    \\
			\hline
			\textbf{Angiv godkendelse} &     \\
			\hline
			\textbf{Initialer} &     \\
			\hline
			\textbf{Dato} &    \\
			\hline
		\end{longtable}
	\end{center}
	
\newpage	
	
%	\begin{center}
%		\begin{longtable}{ | m{4cm}| m{8.5cm}|} 
%			\hline
%			\textbf{Krav nr.} & Undtagelse 3.1   \\ 
%			\hline
%			\textbf{Handling} & Genstart initialseringen af Kameraet  \\
%			\hline
%			\textbf{Forventet resultat} & Feedback fra kameraet om initialisering starter \\
%			\hline
%			\textbf{Testmetode}  & USB kablet til Kameraet frakobles. Resultatet observeres på GUI ved en messagebox.   \\
%			\hline
%			\textbf{Resultat}  &    \\
%			\hline
%			\textbf{Angiv godkendelse} &     \\
%			\hline
%			\textbf{Initialer} &     \\
%			\hline
%			\textbf{Dato} &    \\
%			\hline
%		\end{longtable}
%	\end{center}
	
	\begin{center}
		\begin{longtable}{ | m{4cm}| m{8.5cm}|} 
			\hline
			\textbf{Krav nr.} & Undtagelse 3   \\ 
			\hline
			\textbf{Handling} & Kameraet initialiseres ikke  \\
			\hline
			\textbf{Forventet resultat} & System-besked: Kameraet kunne ikke initialiseres. Kontrollér forbindelse. Vil du prøve igen? \\
			\hline
			\textbf{Testmetode}  & USB kablet til kameraet frakobles før der trykkes [Start]. Resultatet observeres på GUI ved en dialogboks   \\
			\hline
			\textbf{Resultat}  &    \\
			\hline
			\textbf{Angiv godkendelse} &     \\
			\hline
			\textbf{Initialer} &     \\
			\hline
			\textbf{Dato} &    \\
			\hline
		\end{longtable}
	\end{center}
\newpage
 \subsection{Use case 2: Sortering af langerhanske øer}

	\begin{center}
		\begin{longtable}{ | m{4cm}| m{8.5cm}|} 
			\hline
			\textbf{Krav nr.} & 2.1  \\ 
			\hline
			\textbf{Handling} & Systemet detekterer en Langerhansk ø.  \\
			\hline
			\textbf{Forventet resultat} & Tælleren for antal sorterede øer stiger \\
			\hline
			\textbf{Testmetode}  & Sorteringscyklussen er startet. Det observeres på GUI, at tælleren for antal detekterede øer stiger.   \\
			\hline
			\textbf{Resultat}  &    \\
			\hline
			\textbf{Angiv godkendelse} &     \\
			\hline
			\textbf{Initialer} &     \\
			\hline
			\textbf{Dato} &    \\
			\hline
		\end{longtable}
	\end{center}	
	
	\begin{center}
		\begin{longtable}{ | m{4cm}| m{8.5cm}|} 
			\hline
			\textbf{Krav nr.} & 2.2 \& 2.3  \\ 
			\hline
			\textbf{Handling} & Arduino sender signal til ventilen om åbning.  \\
			\hline
			\textbf{Forventet resultat} & Ventilen er åben \\
			\hline
			\textbf{Testmetode}  & Observeres ved at se, at der løber væske ned i beholderen med isolerede øer  \\
			\hline
			\textbf{Resultat}  &    \\
			\hline
			\textbf{Angiv godkendelse} &     \\
			\hline
			\textbf{Initialer} &     \\
			\hline
			\textbf{Dato} &    \\
			\hline
		\end{longtable}
	\end{center}	
	
	

	\begin{center}
		\begin{longtable}{ | m{4cm}| m{8.5cm}|} 
			\hline
			\textbf{Krav nr.} & 2.4 \& 2.5  \\ 
			\hline
			\textbf{Handling} & Arduino sender signal til ventilen om lukning.   \\
			\hline
			\textbf{Forventet resultat} & Ventilen er lukket \\
			\hline
			\textbf{Testmetode}  & Observeres ved at se væsken løber ud i wastebeholderen  \\
			\hline
			\textbf{Resultat}  &    \\
			\hline
			\textbf{Angiv godkendelse} &     \\
			\hline
			\textbf{Initialer} &     \\
			\hline
			\textbf{Dato} &    \\
			\hline
		\end{longtable}
	\end{center}	
	

  \subsection{Use Case 3: Stop sorteringscyklus}
\begin{center}
		\begin{longtable}{ | m{1.785cm} | m{1.785cm}| m{1.785cm}| m{1.785cm}| m{1.785cm}| m{1.785cm}|m{1.785cm}| } 
			\hline
			\textbf{Krav nr.} &\textbf{ Handling} & \textbf{Forventet resultat} & \textbf{Test-metode} &\textbf{Resultat} & \textbf{ \checkmark \textbackslash -} & \textbf{Initialer og dato} \\ 
			
			\hline
			3.1 &  Operatør stopper sorteringscyklussen ved at trykke på [Stop] & Sorteringscyklussen stopper & En sorteringscyklus er i gang. Knappen [Stop] trykkes. Resultatet observeres på GUI.  &  & & \\
			\hline
			
			\hline
			3.2 &  Systemet slukker for pumpen & Flowet i slangen stopper & Observeres ved at se på flowet i slangen &  & & \\
			\hline
			
			3.3 &  Systemet slukker for kameraet  & Kameraets sluk signal modtages og gives til UI & Det observeres på GUI’en at kameraet er slukket.  &  & & \\
			\hline
			
			3.4 &  Systemet slukker for kamera lyset  & Lyset slukker & Kamera lyset observeres  &  & & \\
			\hline
			
			3.5 &  Systemet slukker for arduinoen & Arduino sluk signal modtages og gives til UI & Det observeres på GUI’en at Arduinonen er slukket. &  & & \\
			\hline
			
			Undtagelse 1 &  1. Celle-opløsnings-beholderen løber tør for væske & Sorterings-cyklussen stopper & En sorteringscyklus er i gang. Sorterings-cyklussen forsættes indtil celleopløsningsbeholderen løber tør for væske. Resultatet observeres på GUI. & & & \\
			\hline
			
	
			Undtagelse 1 &  2. Systemet slukker for pumpen & Flowet i slangen stopper & Observeres ved at se på flowet i slangen &  & & \\
			\hline
			
			Undtagelse 1 &  3. Systemet slukker for kameraet  & Kameraets sluk signal modtages og gives til UI & Det observeres på GUI’en at kameraet er slukket.  &  & & \\
			\hline
			
			Undtagelse 1 &  4. Systemet slukker for kamera lyset  & Lyset slukker & Kamera lyset observeres  &  & & \\
			\hline
			
			Undtagelse 1 &  5. Systemet slukker for arduinoen & Arduino sluk signal modtages og gives til UI & Det observeres på GUI’en at Arduinonen er slukket. &  & & \\
			\hline			
			
		\end{longtable}
		
	\end{center}
	\pagebreak
 \subsection{Test setup use case 4: Indstillinger}
\begin{center}
		\begin{longtable}{ | m{1.785cm} | m{1.785cm}| m{1.785cm}| m{1.785cm}| m{1.785cm}| m{1.785cm}|m{1.785cm}| } 
			\hline
			\textbf{Krav nr.} &\textbf{ Handling} & \textbf{Forventet resultat} & \textbf{Test-metode} &\textbf{Resultat} & \textbf{ \checkmark \textbackslash -} & \textbf{Initialer og dato} \\ 
			
			\hline
			4.1 &  Operatøren klikker på [Indstilligner]. & Et nyt vindue åbner med systemets indstillinger. & Knappen [Indstillinger] trykkes.  &  & & \\
			\hline
			
			
			4.2 \& 4.3 \fxnote{er det ok?} &  Arduino sender signal til ventilen om åbning. & Udgang til ventilen er høj & Pin D7 måles vha. multimeter. &  & & \\
			\hline
			
			Undtagelse 1 &  Operatøren klikker [Annuller].  & Indstillings-vinduet lukkes og systemets indstillinger er uændret. & Knappen [Annuller] trykkes. Det verificeres at indstillingerne er uændret ved at åbne Indstillinger igen. \fxnote{bør denne uddybes?}  &  & & \\
			\hline
			

			
			
		\end{longtable}
		
	\end{center}
	\pagebreak
\newpage
  \subsection{Use case 5: Data logning}

	\begin{center}
		\begin{longtable}{ | m{4cm}| m{8.5cm}|} 
			\hline
			\textbf{Krav nr.} & 5.1 \\ 
			\hline
			\textbf{Handling} & Systemet gemmer en fil i formatet .csv for sorteringscyklussen  \\
			\hline
			\textbf{Forventet resultat} & Filen er gemt i databasen  \\
			\hline
			\textbf{Testmetode}  & En ny sorteringscyklus startes (UC 1), hvorefter sorteringscyklussen stoppes (UC 3) ved tryk på [Start] og [Stop]. Det observeres ved styresystemets filsystem, at filen er gemt  \\
			\hline
			\textbf{Resultat}  &    \\
			\hline
			\textbf{Angiv godkendelse} &     \\
			\hline
			\textbf{Initialer} &     \\
			\hline
			\textbf{Dato} &    \\
			\hline
		\end{longtable}
	\end{center}			

	\begin{center}
		\begin{longtable}{ | m{4cm}| m{8.5cm}|} 
			\hline
			\textbf{Krav nr.} & 5.2 \\ 
			\hline
			\textbf{Handling} & Systemet informerer operatøren om, at filen er gemt.  \\
			\hline
			\textbf{Forventet resultat} & Der vises besked til operatøren.  \\
			\hline
			\textbf{Testmetode}  & En ny sorteringscyklus startes (UC 1), hvorefter sorteringscyklussen stoppes (UC 3) ved tryk på [Stop]. Observer GUI for messagebox.  \\
			\hline
			\textbf{Resultat}  &    \\
			\hline
			\textbf{Angiv godkendelse} &     \\
			\hline
			\textbf{Initialer} &     \\
			\hline
			\textbf{Dato} &    \\
			\hline
		\end{longtable}
	\end{center}