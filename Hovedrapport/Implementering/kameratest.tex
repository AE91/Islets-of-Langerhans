\subsubsection{Test}
Det indkøbte kamera er testet ved at tage en serie af billeder af langerhanske øer. Billedserien skulle i første omgang danne grundlag for den videre billedbehandling. Første test bestod af 107 billeder taget af langerhanske øer, herunder billeder kun af isolerede øer og enkelte baggrundsbilleder. Forsøgsopstillingen var en efterligning af den nuværende sorteringsproces, hvor opløsningen hældes i petriskåle. Petriskålen placeres på en sort bagplade, hvorefter operatøren isolerer øerne ved at kigge gennem et mikroskop. I forsøgsopstillingen anvendtes der i stedet for mikroskopet det indkøbte kamera. En række billeder blev udvalgt til yderligere analyse, hvor Søren Gregersen udpegede hvilke elementer der var øer. Efter nærmere analyse af billederne viste det at det indkøbte kamera ikke var tilstrækkelig til at lave segmentering på billederne. De punkter, hvor kameraet ikke var tilstrækkelig kvalitet var bl.a. autoeksponering og autofokus.

Når øerne blev observeret gennem et almindeligt mikroskop var der langt større kontrast i mellem øerne og det ekstra væv. På billederne fra kamerat var der ikke denne forskel, hvilket udelukkede segmentering på denne parameter. Herudover var det tydeligt, at billederne ikke var skarpe nok, hvilket gjorde at det var svært at vurdere hvad der var øer og ikke øer.

Herudover er der er en polariserende effekt på billederne, hvor nogle af elementerne lyser meget kraftigt op grundet belysningen. Dette gør det er svært at vurdere strukturer og størrelser på elementerne i billedet. 

Efter den første test blev det i samarbejde med vejleder og kunde besluttet, at det indkøbte kamera ikke er af tilstrækkelig kvalitet til at detektere langerhanske øer. I dette system bliver der i stedet udviklet et sæt af billeder, som skal simulerer flowet i slangerne. Billederne skal indeholde langerhanske øer, ekstra væv og tilfældig støj. Udfra disse billeder skal langerhanske øer detekteres, hvorefter systemet skal åbne ventilen. Billedesættet skal ses som en simulering af kameraet. Implementeringen af denne funktionalitet er nærmere beskrevet under afsnit REF. I en senere prototype vil et kamera af højere kvalitet være påkrævet. Her er det især krav til autoeksponering, autofokus og depth of view som er essentielle. Herudover vil et kamera med polariseringsfilter være oplagt til at fjerne evt. genskær fra belysningen. Hos Farnell er der andre producenter end det indkøbte Duratool mikroskop, som bl.a. har polariseringsfilter. En test af forskellige kameraer ville være oplagt til finde det ideelle kamera til optagelse af langerhanske øer. 