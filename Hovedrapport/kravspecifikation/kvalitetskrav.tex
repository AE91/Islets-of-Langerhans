\subsection{Kvalitetskrav}
\label{subsec:Kvalitetskrav}
Systemet har følgende krav fra kunden
\begin{center}
		\begin{longtable}{ | m{0.5cm} | m{3cm}| m{6cm}| m{4cm} |} 
			\hline
			\textbf{Nr} & \textbf{Krav} & \textbf{Beskrivelse} & \textbf{Kommentar} \\ 
			\hline
			1 & Hastighed & Hastigheden på systemet skal være højere end 30 øer sorteret pr. minut & \\
			\hline
			2 & Renhed & 2.1 mere end 90 \% af de isolerede øer skal være faktiske øer 
(Sandt pos: > 90 \%)

2.2 der skal være mindre end 5 \% af de isolerede øer, der ikke er øer
(Falsk pos: < 5 \%)

2.3 der skal være mindre end 5 \% af øerne i opløsningen der ikke er blevet isoleret
(Falsk neg: < 5 \%)
 & Dokumentation af renhed:
 \begin{enumerate}
 \item Vurdering af erfaren ø-plukker.
 \item Opmåling v.hj.a. digital billedbehandlingssoftware (ref 1 \fxnote{INDSÆT REFERENCER}).
  \item Funktionstests i laboratoriet (ref 1 og 2 ).
 \end{enumerate} \\
			\hline
			3 & Isoleringsgrad & Over 90 \% af det oprindelige antal, skal være isoleret &
			 $\frac{\text{Antal isolerede}}{\text{Total antal i opløsning}} * 100$\\
			\hline
			4 & Genkendelsesgrad & Over 90 \% af det oprindelige antal, skal være isoleret &
			 $\frac{\text{Visionsgenkendte}}{\text{Total antal i opløsning}} * 100$\\
			\hline
			5 & Ø/Cellestørrelse (\SI{}{\micro\metre}) & Systemet skal minimum kunne sortere øer, der 
har en størrelse mellem \SI{100}{\micro\metre} og \SI{300}{\micro\metre}
 &		\\
			\hline
			6 & Datalogning & Systemet skal kunne logge informationer omkring opløsningens øer, både størrelse og form & \\
			\hline
			7 & Rensning & Systemet skal kunne lave en automatisk rensning af rør mm. & \\
			\hline
			8 & Køling & Systemet skal kunne køle opløsningensvæsken. & \\
			\hline
		\end{longtable}
		
		
		
	\end{center}
	\pagebreak