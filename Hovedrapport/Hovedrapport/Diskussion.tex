\chapter{Diskussion}

\section{Billedprocessering}
Da den udviklede prototype segmenterer øer ud fra billeder, der er genereret, kan det ikke konkluderes, at detektionsalgoritmen vil være lige så effektiv ved rigtige øer. Algoritmen vil givetvis skulle indeholde flere trin i segmenteringen for at give en præcis detektering af øerne. Dette vil gå ud over den tid det tager, at processere de enkelte billeder, som vil give en lavere framerate. En lavere framerate kan have betydning for, hvorvidt en ø detekeres eller ej, da der mistes noget information. Dette vil have betydning for om en ø bliver isoleret af ventilen. Flow hastigheden i væsken kan godt sænkes, men dermed opfyldes kravet om 30 øer i minuttet ikke.

Herudover vil en anden mængde eksokrint væv fra opløsningsvæsken end den fra de genererede billeder kunne bidrage med yderligere støj. Dette vil potentielt kunne påvirke præcissionen af detektionsalgoritmen og derved detektere flere øer end der reelt er. 

Ideelt set skulle billedprocesseringen være baseret på billeder af øerne gennem slangen, men grundet den første test af kameraet blev det klart at kvaliteten af billederne ikke havde tilstrækkelig kvalitet. Derfor blev billedprocesseringen baseret på de genererede billeder. 

\section{Ventil og simuleringsvæske}
Prototypen har vist, at det ikke er muligt at frasortere objekter fra simuleringsvæsken vha. ventilen. Det skal undersøges nærmere hvorfor ventilen blokkerer for objekterne selvom ventilens diameter er specificeret til 1 mm. Enhedstesten af ventilen viste, at Matlab timerne er præcise i forhold til styring af hvornår ventilen skal åbne og lukke. Tidsforsinkelserne er dog baseret på udregninger, og derfor skal det optimeres hvordan øernes væskeegenskaber har indvirkning for hvornår ventilen skal åbne og lukke. Herudover skal tiden optimeres i en endelig prototype for, at minimere mængden af eksokrint væv som også bliver isoleret. En anden vigtig tidsfaktor er, at tallene fra ventilens åbnings- og lukningstider er den maksimale tid det vil tage. Derfor kan der komme eksokrin væv med hvis ventilen åbner hurtigere. Ligeledes kan en langerhanske ø blive fanget, inde i ventilen hvis den lukker hurtigere. Til sidst er slangen til beholderen med isoleret øer ikke regnet med, det betyder at en ø bliver i slangen hvis denne er for lang. Derfor vil det være at foretrække at slangen hertil er så kort som mulig.



\section{Cost-benefit analyse}
Resultatet af cost-benefit analysen viser, at der er økonomiske fordele ved at anvende en automatiseret løsning til isolering af øerne. Da det ikke har været muligt, at lave en egentlig test af prototypen til sortering af øerne er en række af tallene estimeret. Derfor vil resultatet af cost-benefit analysen givetvis være anderledes ved en egentlig implementering af systemet. Herudover er systemts profit til dækning af ekstra udgifter sat til 400 \%, hvilket ligeledes er et estimat af den udgift der vil være til ekstra udgifter. Yderligere er tilbagebetalingsperioden på 92 uger baseret på, at der laves et batch af øer pr. uge. I perioder hvor der foretages flere batch pr. uge vil tilbagebetalingsperioden dermed forkortes ved investering af en automatiseret løsning.

En af de ting der ikke er belyst i cost-benefit analysen er om en automatiseret løsning vil have en betydning for arbejdsmiljøet. Da der bruges 6 timer på sorteringsprocessen ved den manuelle metode vil den automatiserede løsning potentielt kunne forbedre arbejdsmiljøet. Det kan være svært, at svare på i hvilken grad arbejdsstillingen ved den manuelle metode medfører arbejdsskader. Hvor en automatiseret løsning vil kunne forbedre arbejdsmiljøet er i højere grad ved, at frigøre den tid den PhD studerende bruger på sorteringsprocessen til eksempelvis, at forberede videnskabelige forsøg.

\section{Hardware}
Resultatet af hardwaren er mundet ud et shield til Arduino platformen, i stedet kunne printet indeholde en microcontroller. Funktionen af dette vil gøre arduinoen udnødvendig, men det vil kræve en bootloader til microcontroller for at kunne virke med \textit{Matlabs} arduino pakke. I samme ombæring kan det overvejes om pakken skal laves om så der, sendes en string i gennem den serielle forbindelse i stedet. Dette vil skabe mere kontrol over Arduinoen, dog er det en stor arbejdsopgave på noget der allerede fungerer. Økonomisk vil der ikke være meget, at indtjene ved at integrere microcontrolleren i printet fordi Arduino unoens favorable pris. Produktet vil derimod opnå et mere professionelt udseende, end ved brug af Arduino platformen.

\section{Accepttest}
Den udførte accepttest blev ikke godkendt, da kammeraet ikke levede op til den ønskede kvalitet. Kameraet blev derfor i prototypen erstattet af en funktion i Matlab, som simulerede kameraet. Derfor var det ikke muligt, at teste kvalitetskravene i accepttesten, da detektionen ikke blev foretaget på rigtige øer. Dermed kunne det heller ikke testes om ventilen ville frasortere øerne som beskrevet i kravspecifikationen. I en endelig prototype, hvor et kamera er implementeret ville en test af disse krav skulle testes igen for at verificere, at systemet lever op til de specificerede kvalitetskrav. 