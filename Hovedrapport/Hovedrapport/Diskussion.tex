\chapter{Diskussion}

\section{Billedprocessering}
I og med at den udviklede prototype segmenterer øer ud fra billeder der er genereret kan det ikke konkluderes, at detektionsalgoritmen vil være lige så effektiv ved rigtige øer. Algoritmen vil givetvis skulle indeholde flere trin i segmenteringen for at give en præcis detektering af øerne. Dette vil gå ud over den tid det tager, at processere de enkelte billeder, som vil give en lavere framerate. En lavere framerate kan have betydning for, hvorvidt en ø detekeres eller ej, i det man mister noget information. Dette vil have betydning for om en ø bliver isoleret af ventilen.

Herudover vil en anden mængde eksokrint væv fra opløsningsvæsken end den fra de genererede billeder kunne bidrage med yderligere støj. Dette vil potentielt kunne påvirke præcissionen af detektionsalgoritmen og derved detektere flere øer end der reelt er. 

Ideelt set skulle billedprocesseringen være baseret på billeder af øerne gennem slangen, men grundet den første test af kameraet blev det klart at kvaliteten af billederne ikke var tilstrækkelig kvalitet. Derfor blev billedprocesseringen baseret på de genererede billeder. 
\section{Simuleringsvæske}
Prototypen har vist, at det er muligt at frasortere objekter fra simuleringsvæsken vha. ventilen. Da frøene i simuleringsvæsken har andre væskeegenskaber end øerne, vil der i en endelig prototype skulle optimeres i forhold til den tid der går fra en ø er detekteret til ventilen åbner og lukker igen. Yderligere indeholder simuleringsvæsken ikke uønsket væv, som den egentlige opløsningsvæske gør. Dette er en faktor der skal tages højde for i en endelig prototype, så man minimere mængden af det eksokrine væv som også bliver isoleret.


\section{Cost-benefit analyse}
Resultatet af cost-benefit analysen viser, at der er økonomiske fordele ved at anvende en automatiseret løsning til isolering af øerne. Da det ikke har været muligt, at lave en egentlig test af prototypen til sortering af øerne er en række af tallene estimeret. Derfor vil resultatet af cost-benefit analysen givetvis være anderledes ved en egentlig implementering af systemet. Herudover er systemts profit til dækning af ekstra udgifter sat til 400 \%, hvilket ligeledes er et estimat af den udgift der vil være til ekstra udgifter. Yderligere er tilbagebetalingsperioden på 92 uger baseret på, at der laves et batch af øer pr. uge. I perioder bliver hvor der foretages flere batch pr. uge vil tilbagebetalingsperioden dermed forkortes ved investering af en automatiseret løsning.

En af de ting der ikke er belyst i cost-benefit analysen er om en automatiseret løsning vil have en betydning for arbejdsmiljøet. Da der bruges 6 timer på sorteringsprocessen ved den manuelle metode vil den automatiserede løsning potentielt kunne forbedre arbejdsmiljøet også. Det kan være svært, at svare på i hvilken grad arbejdsstillingen ved den manuelle metode medfører arbejdsskader. Der hvor en automatiseret løsning vil kunne forbedre arbejdsmiljøet er i højere grad ved, at frigøre den tid den PhD studerende bruger på sorteringsprocessen til eksempelvis, at forberede videnskabelige forsøg.

\section{Accepttest}
Den udførte accepttest blev ikke godkendt, da kammeraet ikke levede op til den ønskede kvalitet. Kameraet blev derfor i prototypen erstattet af en funktion i Matlab, som simulerede kameraet. Derfor var det ikke muligt, at teste kvalitetskravene i accepttesten da detektionen ikke blev foretaget på rigtige øer. Dermed kunne det heller ikke testes om ventilen ville frasortere øerne som beskrevet i kravspecifikationen. I en endelig prototype, hvor et kamera er implementeret ville en test af disse krav skulle testes igen for at verificere at systemet lever op til de specificerede kvalitetskrav. 