\chapter{Diskussion}

\section{Billedprocessering}
I og med at den udviklede prototype segmenterer øer ud fra billeder der er genereret kan det ikke konkluderes, at detektionsalgoritmen vil være lige så effektiv ved rigtige øer. Algoritmen vil givetvis skulle indeholde flere trin i segmenteringen for at give en præcis detektering af øerne. Dette vil gå ud over den tid det tager, at processere de enkelte billeder, som vil give en lavere framerate. En lavere framerate kan have betydning for, hvorvidt en ø detekeres eller ej, i det man mister noget information. Dette vil have betydning for om en ø bliver isoleret af ventilen.

Herudover vil en anden mængde exokrint væv fra opløsningsvæsken end den fra de genererede billeder kunne bidrage med yderligere støj. Dette vil potentielt kunne påvirke præcissionen af detektionsalgoritmen og derved detektere flere øer end der reelt er. 
\section{Simuleringsvæske}
Prototypen har vist, at det er muligt at frasortere objekter fra simuleringsvæsken vha. ventilen. Da frøene i simuleringsvæsken har andre væskeegenskaber end øerne, vil der i en endelig prototype skulle optimeres i forhold til den tid der går fra en ø er detekteret til ventilen åbner og lukker igen. Yderligere indeholder simuleringsvæsken ikke uønsket væv, som den egentlige opløsningsvæske gør. Dette er en faktor der skal tages højde for i en endelig prototype, så man minimere mængden af det eksokrine væv som også bliver isoleret.

\section{Kamera}
\textbf{Ved ikke lige om det her afsnit er mere til perspektiveringen??}

Testen af det indkøbte kamera viste en række problemstillinger, som der skal tages højde for ved indkøb af et kamera af bedre kvalitet. Det vil være oplagt, at foretage en test af en række kameraer
\section{Cost-benefit analyse}
Resultatet af cost-benefit analysen viser, at der både er økonomiske og kvalitative fordele ved at anvende en automatiseret løsning til isolering af øerne. Da det ikke har været muligt, at lave en egentlig test af sorteringshastigheden for den udviklede prototype, vil det økonomiske resultat potentielt være lavere. Det er antaget i cost-benefit analysen, at systemet kan sortere 30 øer pr. minut. Denne hastighed vil skulle testes ved en endelig implementering, for at give et mere præcist resultat af cost-benefit analysen. 


Diskussion af resultater

- punkter?

vi burde have søgt midler til et bedre kamera fra starten. Det ville have optimalt at have testet kameraet med øerne inde i slangen. 

