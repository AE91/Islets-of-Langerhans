\chapter{Perspektivering}
\section{Kamera}
Det indkøbte kamera viste sig ikke at være af tilstrækkelig kvalitet. Det blev klart under en enhedstest, hvor der blev tager billeder af de langerhanske øer i colleganeseopløsningen placeret i en petriskål. Kameraet fangede ikke den højere lysintensitet ved de langerhanske øer, hvilket gjorde at rest vævet lignede øerne. Yderligere blev det erfaret at lyset er utrolig vigtigt, derfor er ideen kamerahuset en nødvendig til at kontroller lyset til kameraet. I videre udvikling af projektet bør der investeres i et  bedre kamera med manuel fokus justering, da det er erfaret at autofokus ikke er at foretrække. 

\section{Pumpe}
- sprøjtepumpe: + minimal påvirkning af øerne. - ikke kontinuert flow, lavt flow
\section{Påvirkning af øer}
I et senere stadie af projektet bør der undersøges, hvor meget tryk og stress de langerhanske øer kan tåle. Der er mange faktorer der stadig er ukendte hvilket bør undersøges, det kræves derfor at bruges tid på at finde ud af dette. Metoden det kan gøres på er at teste hvor meget insulin øerne producerer efter, at de har været påvirket på forskellige måder. Det skal blandt andet undersøges hvordan øerne opfører sig ved en peristaltisk pumpe, derudover skal der også ventilen på samme måde testes. 

Ved ventilen skal der også undersøges hvordan tid der går fra kameraet har detekteret den til den er ved ventilen. Det er vigtigt at ventilen isolerer de langerhanske øer, men det er heller ikke anvendeligt at få for meget af rest vævet med.

Da kameraet højst sandsynlig har et større synsfelt end slangen, kunne der optimeres på sorteringshastigheden ved at have paralle systemer. Systemerne kunne kører forbi det samme kamera, det kræver dog at billedeprocesseringen er hurtig nok, det skal derfor undersøges.

