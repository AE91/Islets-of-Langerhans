\chapter{Perspektivering}
Dette Kapitel indeholder perspektivering af projektet, som kan udføres i en videre udvikling af projektet.
\section{Kamera}
Det indkøbte kamera viste sig ikke at være af tilstrækkelig kvalitet. Det blev klart under en enhedstest, hvor der blev taget billeder af de langerhanske øer i colleganeseopløsningen placeret i en petriskål. Kameraet fangede ikke den højere lysintensitet ved de langerhanske øer, hvilket gjorde at rest vævet lignede øerne. Yderligere blev det erfaret at lyset er utrolig vigtigt, derfor er ideen med kamerahuset en nødvendighed for at kontroller lyset til kameraet. I videre udvikling af projektet bør der investeres i et bedre kamera med manuel fokus justering, da det er erfaret at autofokus ikke er at foretrække. Se testopstilling mm. i afsnit XX i projektdokumentation. 

\section{Påvirkning af øer}
I et senere stadie af projektet bør der undersøges, hvor meget tryk og stress de langerhanske øer kan tåle. Fordi der er mange faktorer der stadig er ukendte hvilket bør undersøges, der kræves derfor et mindre studie omkring dette. Metoden det kan gøres på er at teste hvor meget insulin øerne producerer efter, at de har været påvirket på forskellige måder. På den måde kan de isolerede øer testes inden og efter at de pumpes i gennem systemet, hvor resultatet sammen lignes. Det skal blandt andet undersøges hvordan øerne opfører sig ved en peristaltisk pumpe som brugt i projektet, derudover bør ventilen også undersøges. Giver testen negative resultater bør systemet deles op, for at teste pumpen og ventilen hver for sig.

\section{Pumpe}
Den peristaltiske pumpe virker ved at sammenpresse slangen ved roterende ruller, hvor slangen udvider sig efter sammenpresningen og trækker væsken til sig. Væsken indspærres af den næste rulle som sammenpresser slangen og drives derfor ud af slangen. Fordelene ved en peristaltisk pumpe er at væsken ikke forurenes, da væsken er isoleret fra pumpen vha. slangen. Derudover skabes der et præcist flow og pumpen kan ikke løbe "tør" da den er selvspædende. Det negative ved en peristaltisk pumpe er at den sammenpresser slangen, hvilket kan risikere at klemme på de langerhanske øer. Til at forhindre at øerne bliver beskadiget kan en sprøjtepumpe overvejes, fordi der bør være mindre påvirkning af øerne. Det vil være favorable hvis flowet er konstant med hen blik på stadig, at opnå kravet om de 30 isoleret øer i minuttet. Derfor kan en tocylindret sprøjtepumpe med tilhørende ventiler bruges for, at holde hastigheden konstant. 

\section{Ventil}
Ved ventilen skal der også undersøges hvor lang tid der går, fra kameraet har detekteret den til den er ved ventilen med en langerhanske ø. Fordi forsøget i projektet kun er lavet med simuleringsvæsken, men sandsynligheden for at de langerhanske øer har samme væskeegenskaber som frøerne brugt i simuleringsvæsken er forholdsvis lav. Det er derfor ikke sikkert at tidsintervallet er det samme. Grunden til at forsøget bør laves er at det er vigtigt at ventilen isolerer de langerhanske øer, men det er heller ikke anvendeligt at få for meget af rest vævet med.

\section{Parallelle systemer}
For at opnå en højere hastighed, kan parallelle rør systemer forbi kameraet overvejes.
Da kameraet sandsynligvis har et større synsfelt end en slange. Derfor kan Systemerne føres forbi det samme kamera, det kræver dog at billedeprocesseringen er hurtig nok.

\section{Billedeprocesseringen}
I projektet er billederne til billdebehandling simuleret, ved at bruge billeder taget af isoleredet øer, hvor der er tilføjet tilfældigt støj og rest væv. I videre udvikling af projektet bør billedeprocesseringen optimeres til de faktiske øer i slangen. \fxnote{kan du måske uddybe esager?}

\section{Omrøring og køling af celleopløsningsbeholderen}
Til videre udvikling af de ikke-elektroniske dele, bør en omrøring i opløsningsvæsken skabes. Det skal der fordi de langerhanske øer bundfælder, samtidigt bør omrøringen håndteres skånsomt så øerne ikke beskadiges. En mulighed kunne være at den tidligere omtalte sprøjtepumpe roteres, hvor der på den måde vil blive skabt en skånsom omrøring af opløsningsvæsken.

