\chapter{Perspektivering}
Dette kapitel indeholder perspektivering af projektet, som kan udføres i en videreudvikling af projektet.
\section{Kamera}
Det indkøbte kamera viste sig i projektet ikke at være  tilstrækkelig kvalitet. I videreudvikling af projektet bør der laves en test af en række kameraer for, at sikre denne del af systemet er tilstrækkelig kvalitet. Dermed kan kvaliteten af en kameraerne testes inden et bestemt kamera indkøbes. Testen skal foretages ved, at man tager en række billeder af langerhanske øer, ideelt hvor øerne pumpes i gennem slangen for at komme tæt på brugscenariet. 

\section{Påvirkning af langerhanske øer}
I et senere stadie af projektet bør der undersøges, hvor meget tryk og stress de langerhanske øer kan tåle. Fordi der er mange faktorer der stadig er ukendte hvilket bør undersøges, der kræves derfor et mindre studie omkring dette. Metoden det kan gøres på er at teste hvor meget insulin øerne producerer efter, at de har været påvirket på forskellige måder. På den måde kan de isolerede øer testes inden og efter at de pumpes i gennem systemet, hvor resultatet sammenlignes. Det skal blandt andet undersøges hvordan øerne opfører sig ved en peristaltisk pumpe som brugt i projektet, derudover bør ventilen også undersøges. Giver testen negative resultater bør systemet deles op, for at teste pumpen og ventilen hver for sig.

\section{Pumpe}
Den peristaltiske pumpe virker ved at sammenpresse slangen ved roterende ruller, hvor slangen udvider sig efter sammenpresningen og trækker væsken til sig. Væsken indespærres af den næste rulle som sammenpresser slangen og drives derfor ud af slangen. Fordelene ved en peristaltisk pumpe er at væsken ikke forurenes, da væsken er isoleret fra pumpen vha. slangen. Derudover skabes der et præcist flow og pumpen kan ikke løbe "tør" da den er selvspædende. Ulempen ved en peristaltisk pumpe er at den netop sammenpresser slangen, hvilket kan risikere at klemme på de langerhanske øer. Til at forhindre at øerne bliver beskadiget kan en sprøjtepumpe overvejes, fordi der bør være mindre påvirkning af øerne. Det vil være favorable hvis flowet er konstant med henblik på stadig, at opnå kravet om de 30 isolerede øer i minuttet. Derfor kan en tocylindret sprøjtepumpe med tilhørende ventiler bruges for, at holde hastigheden konstant. 

\section{Ventil}
Ved ventilen skal der også undersøges hvor lang tid der går, fra kameraet har detekteret den til den er ved ventilen med en langerhanske ø. Fordi forsøget i projektet kun er lavet med simuleringsvæsken, men sandsynligheden for at de langerhanske øer har samme væskeegenskaber som frøerne brugt i simuleringsvæsken er forholdsvis lav. Det er derfor ikke sikkert at tidsintervallet er det samme. Grunden til at forsøget bør laves er at det er vigtigt at ventilen isolerer de langerhanske øer, men det er heller ikke anvendeligt at få for meget eksokrint væv med.

\section{Parallelle systemer}
For at opnå en højere hastighed, kan parallelle rør systemer ført forbi kameraet overvejes.
Da kameraet sandsynligvis har et større synsfelt end en slange. Derfor kan systemerne føres forbi det samme kamera, det kræver dog at billedeprocesseringen er hurtig nok.

\section{Billedprocessering}
I projektet er billederne til billedbehandling simuleret, ved at bruge billeder taget af isoleredet øer, hvor der er tilføjet tilfældigt støj og rest væv. I videreudvikling af projektet bør billedprocesseringen optimeres til de faktiske øer i slangen.

\section{Omrøring og køling af celleopløsningsbeholderen}
Til videreudvikling af de ikke-elektroniske dele, bør en omrøring i opløsningsvæsken skabes, da de langerhanske øer bundfælder. Samtidigt bør omrøringen håndteres så skånsomt som muligt så øerne ikke beskadiges. En mulighed kunne være at den tidligere omtalte sprøjtepumpe roteres, hvor der på den måde vil blive skabt en skånsom omrøring af opløsningsvæsken. Ydermere ville det være at foretrække, hvis systemet kunne nedkøle opløsningsbeholderen, for at sikre at enzymet ikke aktiveres uønsket. Det kunne være bestående af et køle element. 

\section{Medicinsk udstyr}
 Da systemet primært er fokuseret til forskningsbrug, kræver udstyret ikke medicinsk godkendelse. Dog kan systemet på sigt muligvis bruges til ø implantationer, for lindring og som præparat til diabetes. Skal systemet bruges til dette formål, vil det kræve en medicinsk godkendelse. For at et produkt kan opnå medicinsk godkendelse, skal produktet overholde en række direktiver, som kan opnås i gennem harmoniseret standarder. Det første punkt ved medicinsk godkendelse er udstyrets formål (intended use), da det er formålet der styrer klassifikationen for udstyret. I Europa klassificeres der efter 4 klasser I, IIa, IIb og III, hvor klasse III er den med de strengeste krav. Uanset klasse skal udstyret opfylde de væsentlige krav i MDD direktivet, som bla. er rapportering af utilsigtede hændelser og overvågning af sysemet på markedet. Ydermere skal der foretages en risikoanalyse af produktets risikoer. For at et medicinsk udstyr kan CE mærkes skal det alt efter klassificering, gennemgå trin i gennem bilagene i MDD direktivet. Det skal indeholde teknisk dokumentation, væsentlige krav og risikoanalyse. Ydermere skal der laves kvalitetsstyringssystemer til at dokumentere produktets forløb. Dokumentationen skal godkendes af et bemyndiget organ, hvis udstyret vurderes til at være højere end klasse I. Ydermere er der høje krav til versionsstyring af produktet, for at give en gennemsigtig sporbarhed. Før der begyndes på arbejdet med medicinsk godkendelse, er det vigtigt at intended use klarlægges. Det skal det for at være sikker på, om udstyret går ind under MDD direktivet. Det bør undersøges om produktet kan sammenlignes med en insulinpen, da den går ind under det farmaceutisk direktiv. Det gør den fordi det er insulinen som virker. 