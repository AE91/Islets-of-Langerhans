\chapter{Resultater}

\section{Det færdige system}
\subsection{Kamera}

\subsection{Tidsinterval mellem kamera og ventil}




\section{Simuleringsvæske}
\label{sec:simuleringsv}
Da de langerhanske øer ikke er lettilgængelig, har det været nødsaget at finde et objekt til at agere aktør for cellerne. Simuleringsvæsken skal bruges til at teste de mekaniske dele, samt tidsintervallet mellem kameraet og ventilen. Egenskaberne af de langerhanske øer der er brugt til at finde simuleringsvæsken er at de er runde og lyse, størrelsen, samt at de bundfælder i colleganse opløsningen. Der har været to iterationer for at finde en aktør for cellerne, den første har bestået i at finde perfekte runde og hvide objekter med en $massefylde<1$. Til dette blev der brugt polysterene, som er lettilgængelig, samt runde, hvide og kan fåes med en massefylde lige over en \fxnote{reference til bedre side end wiki?}. Dog blev det fundet svært at finde polysterene kugler i størrelsen 0.1-0.3mm, derfor blev der skabt kontakt til forhandlere af flamingo kugler. Dog var de tilsendte kugler for store, samt flød oven på. 

Den anden iteration bestod i at finde et biologisk materiale, hvor ved faktorerne runde og hvide blev nedprioriteret for at finde et objekt. For at finde en celle aktør med lettilgængelighed blev plantefrø den næste mulighed. Trods begrænset data omkring plantefrø, lykkes det gruppen at finde plantefrø i størrelsen 0.1-0.3mm vha. forhandlere af plantefrø og andre eksperter \fxnote{referencer til mail}. De indkøbte frø består af timian frø og rævehale frø \fxnote{indsæt billeder}. Udover de indkøbte frø kan der ved videre bearbejdelse af projektet overvejes transparent testa (tt4 mutant) frø. De skulle i følge \textit{Carsten Meier} være hvide, forholdsvis runde, dog er de ikke lige så tilgængelige, som de allerede indkøbte frø. Dette har været grunden til, at de ikke er brugt i projektet ind til nu. 
 

\section{Cost-benefit analyse}
Økonomisk






Andre ting?