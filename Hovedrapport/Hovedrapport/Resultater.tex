\chapter{Resultater}

\section{Det færdige system}
- Kamera


\section{Koncept analyse}
hmm...
 
\section{Kravspecifikation og accepttest}
\label{subsec:krav}
For at vise at der er brugt de beskrevne metoder ovenfor, er der valgt at tage eksempler med i rapporten. %for at eftervise de fire udviklingsfaser projektet har været i gennem.

\subsection{Aktør beskrivelse}
Systemets primære aktør er operatøren, som står for påfyldning af celler, start og stop af sorteringsprocessen. Operatøren har mulighed for at interagere med systemet via en grafisk brugergrænseflade. Systemets sekundære aktør er kameraet og PC’ens filsystem. Kameraet er systemets interface til detektion af de Langerhanske øer. Filsystemet er hvor der løbende gemmes en log over sorteringsprocessen.

\subsection{Use Case Diagram}
I Use Case diagrammet (figur: \ref{fig:usecase}) er der vist, hvilke use cases systemet \textit{The Cell Collector} består af. Yderligere er det vist, hvilke aktører der initiere de enkelte use cases. På venstre side er systemets primære aktør \textit{operatøren} vist, mens systemets sekundære aktører \textit{kamera} og \textit{database} er placeret i højre side. 

\begin{figure}[H]
	\centering
	\includegraphics[width=1\textwidth]{billeder/UC_CellCollector.pdf}
	\caption{Use Case diagram for The Cell Collector}
	\label{fig:usecase}
\end{figure}
Efter use case diagrammet var færdigt, blev der udarbejdet fully dressed use cases som kan ses på nedenstående tabel. Tabellen beskriver normalforløbet og undtagelser for \textit{Start sorteringscyklus}, som er den første use case på diagrammet. 
\newpage 

\label{uc:1}
\begin{center}
		\begin{longtable}{ | m{4cm} | m{8cm}| } 
			\hline
			Mål & Start sorteringscyklus \\ 
			\hline
			Initiering &  Use casen initieres af operatøren\\
			\hline
			Aktør & 
			Primær: Operatør
			
			 Sekundær: Kamera			  \\ 
			\hline
			Startbetingelser & The Cell Collector programmet er startet på computeren \\ 
			\hline	
			Slutbetingelser ved succes & Systemet starter med sorteringen af Langerhanske øer \\
			\hline
			Slutbetingelser ved undtagelse & N/A \\
			\hline
			Normalforløb & \begin{enumerate}
				%\setlength\itemsep{0cm} % Decrease line distance
				\item Operatør fylder celleopløsningsbeholderen
				\item Celleopløsningsbeholderen er fyldt
				\item Operatør starter sorteringscyklus ved at klikke på [Start]
				\subitem [Undtagelse 1: Wastebeholder er fyldt] 
				\item Systemet initialiserer Arduinoen
				\subitem [Undtagelse 2: Ingen forbindelse til Arduino]
				\item Systemet kontrollerer celleopløsningsbeholderen ved at konvertere spændingen (\SI{}{\volt})  til \SI{}{\milli\litre}, og vise beholderens indhold (\SI{}{\milli\litre}) på GUI
				\item Systemet initialiserer kameraet
				\subitem [Undtagelse 3: Kameraet initialiserer ikke]
				\item Systemet tænder for kamera lyset
				\item Systemet tænder for pumpen
				
			\end{enumerate} \\ 
			\hline
			Undtagelser & [Undtagelse 1: Wastebeholder er fyldt] 
			
			\begin{enumerate}
			\item Systembesked: Tøm venligst Wastebeholder før start
			\item Operatøren trykker “OK”
			\item Systemet fortsætter opstartprocessen
			\end{enumerate} 
			
			[Undtagelse 2: Ingen forbindelse til Arduino]
			
			\begin{enumerate}
			\item 1.	Systembesked: Ingen forbindelse til Arduino, kontrollér forbindelser.
			\end{enumerate} 
	
			[Undtagelse 3: Kameraet initialiseres ikke]
			
			\begin{enumerate}
			\item System fejlmeddelse: Kameraet er ikke initialiseret:
			\item Genstart initialisering af Kameraet
			\end{enumerate} \\
			\hline
		\end{longtable}
	\end{center}

Efter at normalforløbet og undtagelserne er defineret, blev acceptesten lavet for den givne use case. Til hvert punkt i normalforløbet er der forberedt en test, som indeholder et krav nr, handling dvs det der skal gøres for at starte testen. Derudover er der forventet resultat, som skal ske for at testen kan godkendes. Testmetode beskriver hvordan testen skal udføres og hvordan den godkendes.

\begin{center}
		\begin{longtable}{ | m{4cm}| m{8.5cm}|} 
			\hline
			\textbf{Krav nr.} & 1.1 \& 1.2    \\ 
			\hline
			\textbf{Handling} &  Operatør fylder celle-opløsnings-beholderen   \\
			\hline
			\textbf{Forventet resultat} &  Celle opløsningsbeholderen er fyldt  \\
			\hline
			\textbf{Testmetode}  &  Celle opløsningsbeholderen fyldes med væske  \\
			\hline
			\textbf{Resultat}  &    \\
			\hline
			\textbf{Angiv godkendelse} &     \\
			\hline
			\textbf{Initialer} &     \\
			\hline
			\textbf{Dato} &    \\
			\hline
		\end{longtable}
	\end{center}
	
	
	\begin{center}
		\begin{longtable}{ | m{4cm}| m{8.5cm}|} 
			\hline
			\textbf{Krav nr.} & 1.3    \\ 
			\hline
			\textbf{Handling} &  Operatør starter sorteringscyklus ved at klikke på [Start]  \\
			\hline
			\textbf{Forventet resultat} &  Opstarts processen i gang sættes.  \\
			\hline
			\textbf{Testmetode}  & Knappen [Start] trykkes, observeres ved tekstboks på GUI, med teksten \textit{systemet starter op}.   \\
			\hline
			\textbf{Resultat}  &    \\
			\hline
			\textbf{Angiv godkendelse} &     \\
			\hline
			\textbf{Initialer} &     \\
			\hline
			\textbf{Dato} &    \\
			\hline
		\end{longtable}
	\end{center}
	
	\begin{center}
		\begin{longtable}{ | m{4cm}| m{8.5cm}|} 
			\hline
			\textbf{Krav nr.} & 1.4    \\ 
			\hline
			\textbf{Handling} &  Systemet initialisere Arduinoen   \\
			\hline
			\textbf{Forventet resultat} &  Arduino initialiseret signal modtages og gives til GUI  \\
			\hline
			\textbf{Testmetode}  & Det observeres på GUI at Arduinoen er initialiseret, i en tekstboks med testen \textit{Arduino er initialiseret}.   \\
			\hline
			\textbf{Resultat}  &    \\
			\hline
			\textbf{Angiv godkendelse} &     \\
			\hline
			\textbf{Initialer} &     \\
			\hline
			\textbf{Dato} &    \\
			\hline
		\end{longtable}
	\end{center}
	
	\begin{center}
		\begin{longtable}{ | m{4cm}| m{8.5cm}|} 
			\hline
			\textbf{Krav nr.} & 1.5    \\ 
			\hline
			\textbf{Handling} &  Systemet kontrollerer celle-opløsnings-beholderen  \\
			\hline
			\textbf{Forventet resultat} &  Antal ml i celleopløsningsbeholderen vises på GUI.  \\
			\hline
			\textbf{Testmetode}  & Der hældes 100 ml i celleopløsningsbeholderen, det observeres på GUI om der vises 100 ml $\pm$ 10 ml   \\
			\hline			
			\textbf{Resultat}  &    \\
			\hline
			\textbf{Angiv godkendelse} &     \\
			\hline
			\textbf{Initialer} &     \\
			\hline
			\textbf{Dato} &    \\
			\hline
		\end{longtable}
	\end{center}	
	
Resten af acceptesten for use case 1 kan ses i projektdokumentationen \fxnote{hvordan skal der refereres her til ?} 

\section{Design}
\label{subsec:design}
I dette afsnit er der givet et eksempel på hvordan projektet er gået fra krav, til en måder at løse projektet på. Der er blandt andet brugt BDD, IBD, flowchart samt sekvensdiagrammer til at beskrive funktioner og sammenhænge i projektet. 

\subsection{BDD og IBD for hardware}
Nedenstående BDD \ref{fig:bdd_Hardware} giver et overordnet overblik i, hvad \textit{The cell collector} indeholder af hardware elementer. Hierarkiet starter øverst med \textit{The cell collector}, som indeholder tre mindre hardware dele. Styreenheden der er defineret som en Arduino, den indeholder de elementer der bl.a. motordriveren. Motordriveren indeholder yderligere to dele, som den styrer. Det vil sige at pumpen og ventilen bliver styret i gennem motordriveren. Udover styreenheden er der også ikke elektriske dele, som beholdere og førringsveje til opløsningen med langerhanske øer. Den tredje underblok til \textit{The cell collector} er kameraet.

\begin{figure}[H]
	\centering
	\includegraphics[width=1\textwidth]{pdf/BDD_Hardware.pdf}
	\caption{BDD - Cell Collector [Hardware]}
	\label{fig:bdd_Hardware}
\end{figure}

Nedenstående IBD \ref{fig:ibd_Hardware} beskriver mere præcist, hvordan de forskellige komponenter interagerer med hinanden på. Diagrammet er brugt til, at der tidligt i udviklingsforløbet bliver defineret hvilke spændinger og signaltyper systemet skal indeholde. Systemet skal indeholde bestemte typer for, at kunne kommunikere med de interne dele.


\begin{figure}[H]
	\centering
	\includegraphics[width=1\textwidth]{pdf/IBD_Hardware(Arduino).pdf}
	\caption{IBD - Cell Collector [Hardware]}
	\label{fig:ibd_Hardware}
\end{figure}

Til ovenstående diagrammer er der udarbejdet tabeller som beskriver signalerne og de enkelte blokke. \fxnote{reference til projektdokumentation}

Ydermere kan der ses specifikationerne for vægtcellen er stillet op og hvilke overvejelser der er gjort ved denne komponent.
\subsection{Vægtcelle}
\label{subsec:loadcell}
Vægtcellen skal bruges til at kontrollere om, der er væske i celleopløsningsbeholderen.

\textbf{Specifikationer for Vægtcelle[\citet{DH7}]:} 
\begin{center}
		\begin{longtable}{ | m{6.5cm} | m{6.5cm}| } 
			\hline
			\textbf{Specifikation} &\textbf{Værdi} \\ 
			\hline
			\textbf{Max belastning:} & 1 kg \\ 
			\hline
			\textbf{Anbefalet arbejdsspænding} & 3-12V \\ 
			\hline
			\textbf{Output} & 1.0mV/V$\pm$0.15mV/V \\ 
			\hline
		\end{longtable}
\end{center}

Den indkøbte vægtcelle kan veje op til 1 kg, hvilket dækker vægten for celleopløsningsbeholderen på 250ml + beholderens vægt. I design fasen er der også beskrevet et teori afsnit for at dokumenter den opnået viden gruppe har fået, design afsnitter indeholder desuden også beregninger og kredsløbsdiagrammer. Det kan ses i den overstående tabel, at vægtcellens output er i millivolt hvilket har medført til at signalet skulle forstærkes. Dette er gjort vha. en operationsforstærker, for at vise et eksempel er der trukket nedenstående ud fra design afsnittet se projektdokumentation afsnit... \fxnote{reference} for hele afsnittet. 

I databladet \ref{bilag:INA114} til INA114 ses det at den har en CMRR på 115dB, ved et gain på 1000 og en indgangsmodstand på 10G$\Omega$. Forstærkningen kan regnes ud fra formlen i databladet \ref{eq:gainina1}
\begin{align}
 G=(1+\frac{50K\Omega}{R_{G}})
 \label{eq:gainina1}
 \end{align} 
 I dette projekt skal der bruges et gain på $\frac{4,9V}{5mV}=980$, 4,9V for ikke at komme i mætning på arduinoens ADC og 5mV da det er den maksimale spænding vægtcellen kan give, ved 5V forsyning.
 \begin{align}
 R_{G}=\frac{50K\Omega}{980-1}=51\Omega
 \label{eq:gainina2}
 \end{align}
Med et gain på 980 giver en $NY_{Maksimalespænding}=980*5mV=4,9V \pm0,147V$, dvs at der nu er en opløsning på
\begin{align}
 \frac{1000g}{trin}=\frac{1000g}{1024}=0,977g/trin=>0,977*\frac{1024}{5V}=200g/V \pm30g
 \label{eq:gainina3}
 \end{align}
 
 Kredsløbet for INA114 og vægtcellen til arduinoen kan ses på figur \ref{fig:loadcelldiagram}
 
  \begin{figure}[H]
	\centering
	\includegraphics[width=0.9\textwidth]{billeder/Hardware/diagrammer/loadcelldiagram.JPG}
	\caption{Diagram for arduino, INA114 og vægtcelle}
	\label{fig:loadcelldiagram}
\end{figure}

\subsection{Sekvensdiagrammer}
Til sidst i design dokumentet er der lavet sekvensdiagrammer for, at få overblik over de sekventielle dele af systemet til hver use case se figur \ref{fig:sekvendisgr} for et eksempel.
\begin{figure}[H]
	\centering
	\includegraphics[width=1\textwidth]{pdf/UC1_cropped.pdf}
	\caption{Sekvensdiagram for usecase 1}
	\label{fig:sekvendisgr}
\end{figure}


\section{Implementering og enhedstest}
\label{subsec:Implement}
I dette afsnit er der vist et eksempel på hvordan delene er implementeret ved at vise vægtcellen for både hardware og software.

\subsection{Hardware}
Efter at diagrammet og kredsløbet blev udført i design dokumentet kunne vægtcellens kredsløb nu testes på et \textit{fumlebræt} 

Efter at diagrammet er fastlagt, testes forbindelserne nu på et \textit{fumlebræt} se figur \ref{fig:loadcelltest} for test opstilling

  \begin{figure}[H]
	\centering
	\includegraphics[width=0.9\textwidth]{billeder/Hardware/diagrammer/Drawing1.jpg}
	\caption{Test opstilling for vægtcelle}
	\label{fig:loadcelltest}
\end{figure}
De primære dele af test koden har været  \textit{sensorValue = analogRead(A0);} og \textit{Serial.println(sensorValue);}, hvor ved at inputtet på \textit{A0} er læst i \textit{serial monitor} som vist på figur \ref{fig:loadcell_test} ved langsom tømning af beholderen.

\begin{figure}[H]
	\centering
	\includegraphics[width=0.9\textwidth]{billeder/Hardware/diagrammer/loadcellunittestbits.JPG}
	\caption{Værdier fra A0 i \textit{serial monitor}}
	\label{fig:loadcell_test}
\end{figure}

 Se bilag \ref{bilag:TKloadcell} for at se hele koden til testen. Til enhedstesten er der brugt et voltmeter til, at måle udgangsspændingen på INA114 for, at se om arduinoens ADC læste rigtigt. Til at sammenligne med voltmeteret, blev \ref{eq:trintilvolt} brugt til at konvertere ADC'ens bits værdi om til spænding.
 
 \begin{align}
 analogRead(A_0)*\frac{5}{1024}=\text{spænding i volt}
 \label{eq:trintilvolt}
 \end{align}
Testopstilling af vægtcellen ser ud som på \ref{fig:loadcell_mont} med celleopløsningsbeholderen. I softwaren kræves det en kalibrering for at vægtcellen er præcis, dette er implementeret i projektdokumentation \fxnote{reference}.
 
 \begin{figure}[H]
	\centering
	\includegraphics[width=0.5\textwidth]{billeder/Hardware/diagrammer/loadcell_montering.pdf}
	\caption{Illustration af opstilling med vægtcelle og celleopløsningsbeholder}
	\label{fig:loadcell_mont}
\end{figure}

\subsection{Software}
Funktionen til load cellen er implementeret efter beskrivelsen i design dokumentet. Dens funktion er, at konvertere det analoge input (V) til indholdet (ml) i celleopløsningsbeholderen. Dette er implementeres ved en lineær model:
\begin{align}
mL = a*input+b \text{, hvor a er hældningen og b er skæringen med y aksen}
\end{align}
Det analoge input ganges altså med en faktor a plus et offset b for at konvertere spænding til antal ml. Nedenstående tabel viser indgangsspændingen for forskellige mængder i beholderen. Udfra disse data er der lavet en lineær regression for at finde hældningen a og skæringen b.
\begin{center}
		\begin{longtable}{ | m{3cm} | m{3cm}| } 
			\hline
			\textbf{ml i beholder} &\textbf{Analog input} \\ 
			\hline
			 \SI{0}{\milli\litre} & \SI{1.9487}{\volt} \\ 
			\hline
			 \SI{25}{\milli\litre} & \SI{2.0440}{\volt} \\ 
			\hline
			\SI{50}{\milli\litre} & \SI{2.1320}{\volt} \\ 
			\hline
			\SI{75}{\milli\litre} & \SI{2.2297}{\volt} \\ 
			\hline
			\SI{100}{\milli\litre} & \SI{2.3109}{\volt} \\ 
			\hline
			\SI{125}{\milli\litre} & \SI{2.4071}{\volt} \\ 
			\hline
			\SI{150}{\milli\litre} & \SI{2.4961}{\volt} \\ 
			\hline
			\SI{175}{\milli\litre} & \SI{2.5821}{\volt} \\ 
			\hline
			\SI{200}{\milli\litre} & \SI{2.6760}{\volt} \\ 
			\hline
			\SI{225}{\milli\litre} & \SI{2.7654}{\volt} \\ 
			\hline
			\SI{250}{\milli\litre} & \SI{2.8587}{\volt} \\ 
			\hline
			\caption{Kalibreringsdata for loadcellen}
			 		\end{longtable}
\end{center}

I Matlab er funktionen \textit{fitlm} anvendt til at finde det bedste lineære fit. Regressionen er baseret på Least Square metoden. \fxnote{Henvisning}
Udfra beregningerne i Matlab er hældningen a og skæringen b fundet til hhv:
\begin{align}
a = 276.14
\end{align}
\begin{align}
b = -539.02
\end{align}
Den endelige funktion er dermed givet ved:
\begin{align}
ml = 276.140*input-539.02
\end{align}

Figur \ref{fig:loadcellcalib} viser den lineære funktion, samt de enkelte data punkter fra tabellen. 
\begin{figure}[H]
	\centering
	\includegraphics[width=0.6\textwidth]{billeder/software/calibration-crop.pdf}
	\caption{Kalibrering af load cell}
	\label{fig:loadcellcalib}
\end{figure}

For at reducere støj og mindske følsomheden overfor hurtigere ændringer i indgangsspændingen er der implementeret en midling af de seneste 10 målinger. Dette er med til, at gøre konverteringen mere robust overfor støj. 

\subsection{Simuleringsvæske}
Da de Langerhanske øer ikke er lettilgængelig, har det været nødsaget at finde en aktør for cellerne. Simuleringsvæsken skal bruges til at teste de mekaniske dele, samt tidsintervallet mellem kameraet og ventilen. Egenskaberne af de langerhanske øer der er brugt til at finde simuleringsvæsken er at de er runde og lyse, størrelsen, samt at de bundfælder i colleganse opløsningen. Der har været to iterationer for at finde en aktør for cellerne, den første har bestået i at finde perfekte runde og hvide objekter med en massefylde over en. Til dette blev der brugt polysterene, som er lettilgængelig, samt runde, hvide og kan fåes med en massefylde lige over en \fxnote{reference til bedre side end wiki?}. Dog blev det fundet svært at finde polysterene kugler i størrelsen 0.1-0.3mm, derfor blev der skabt kontakt til forhandlere af flamingo kugler. Dog var de tilsendte kugler for store, samt flød oven på. 

Den anden iteration bestod i at finde et biologisk materiale, hvor ved faktorerne runde og hvide blev nedprioriteret for at finde et objekt. For at finde en celle aktør med lettilgængelighed blev plantefrø den næste mulighed. Trods begrænset data omkring plantefrø, lykkes det gruppen at finde plantefrø i størrelsen 0.1-0.3mm vha. forhandlere af plantefrø og andre eksperter \fxnote{referencer til mail}. De indkøbte frø består af timian frø og rævehale frø \fxnote{indsæt billeder}. Udover de indkøbte frø kan der ved videre bearbejdelse af projektet overvejes transparent testa (tt4 mutant) frø. De skulle være hvide, forholdsvis runde, dog er de ikke lige så tilgængelige, som de allerede indkøbte frø. Dette har været grunden til at de ikke er brugt i projektet ind til nu. 
 

\section{Cost-benefit analyse}
Økonomisk






Andre ting?