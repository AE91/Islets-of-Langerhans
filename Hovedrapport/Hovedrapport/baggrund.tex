\section{Baggrund}
Til at undersøge sygdommen nærmere og øge forståelsen for de mekanismer der styre insulin reguleringen laves der videnskabelige forsøg med Langerhanske øer.

De øer der anvendes til videnskabelige forsøg stammer typisk fra mus eller rotter. Sorteringsprocessen er foregår over 3 faser, pancreas perfusion, pancreas digestion og til sidst ø isolering. Pancreas perfusionen foregår ved at man operativt fjerner pancreas, hvorefter vævet opløses vha. enzymet collagenase. Enzymet sprøjtes ind i pancreas, hvor en nedbrydning af vævet igangsættes. Collagenasen har den egenskab, at den ikke nedbryder de Langerhanske øer, men kun det  I anden fase sker pancreas digestionen \fxnote{Få protokol af SG} (Kort beskrivelse af faser: pancreas perfusion, pancreas digestion, islet purification).

I sidste fase sker selve ø isoleringen. Der findes en række forskellige metoder til dette, hvor den mest anvendte metode foregår ved manuel isolering af øerne fra en petriskål vha. et mikroskop. Der findes automatiserede isoleringsmetoder bygget på forskellige teknikker herunder gradient baseret centrifugering. Fælles for de nuværende teknikker til automatisk isolering er at der er stor risiko for skade på øerne.

Denne manuelle metode har yderligere ulemper i det den både er besværlig og tidskrævende. Herudover kan der være stor variation i kvaliteten af isoleringen, da der er forskel på den enkelte operatørs håndtering af øerne. 

Derfor ønskes der en automatiseret metode til isolering af langerhanske øer. En automatiseret metode vil kunne give følgende muligheder: 

\begin{itemize}
\item Øget sorteringshastighed for højere udbytte
\item Reducere variation i de isolerede øer
\item Reducere omkostningerne
\item Sikre dokumentation
\end{itemize} \fxnote{Reference til SG powerpoint? Specifikt nok?}

En automatiseret løsning vil potentielt åbne op for nye muligheder indenfor anvendelse af langerhanske øer. Der er 
% kilder om transplantation af øer


 
 
 find videnskabelige artikler der bekræfter at denne metode med at skyde langerhanske øer ind i sukkersyge mus/rotter virker og det er derfor dette projekt er mega relevant og pisse godt
 
 derfor
 
 - videnskabelig artikel over sorteringsprocessen
 
 - videnskabelig artikel over at metoden virker, evt. hvorfor den ikke er mere brugt(sorteringsprocessen er for langsommelig)
 
 
 
 
 
 
 - indsæt systembeskrivelse, så vi konkret får forklaret hvad vores er
 
 \subsection{cellesorteringsprocess}
 \subsection{tidligere systemer}
 Tidligere er der udviklet en prototype til automatiseret ø isolation, der fungerede ved hjælp af kamera detektion og en pipette som sugede den detekterede ø op. Prototypen var dog ikke præcis nok, primært pga. bevægelse i væsken. 
 
 A robotic device was previously developed to pick the islets from the suspension in a petri dish but the accuracy was inadequate partly due to movements of the pipette in the fluid.
