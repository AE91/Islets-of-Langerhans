\section{Baggrund}
For at undersøge sygdommen nærmere og øge forståelsen for de mekanismer der styre insulin reguleringen i kroppen laves der videnskabelige forsøg med Langerhanske øer.

De øer der anvendes til videnskabelige forsøg stammer typisk fra mus eller rotter. Sortering- og isoleringsprocessen foregår over 3 faser (for en nærmere beskrivelse se afsnit \ref{subsec:sortproces}). Kort fortalt sker sorteringen ved at man operativt fjerner pankreas, hvorefter vævet opløses vha. enzymet collagenase. Enzymet sprøjtes ind i pancreas, hvor en nedbrydning af vævet igangsættes. Collagenase har den egenskab, at den ikke nedbryder de Langerhanske øer. Efter nedbrydningen af pankreas sker selve ø isoleringen. Der findes en række forskellige metoder til dette, hvor den mest anvendte metode foregår ved manuel isolering af øerne fra en petriskål vha. et mikroskop. Der findes automatiserede isoleringsmetoder bygget på forskellige teknikker herunder gradient baseret centrifugering. Fælles for de anvendte teknikker til  isolering af øerne er, at der er stor risiko for skade på øerne.

Denne manuelle metode har yderligere ulemper i det, den både er besværlig og tidskrævende. Herudover kan der være stor variation i kvaliteten af isoleringen, da der kan være forskel på den enkelte operatørs håndtering af øerne. 

Derfor ønskes der en automatiseret metode til isolering af langerhanske øer, som minimerer risikoen for skader på øerne. En automatiseret metode vil have følgende følgende fordele: 

\begin{itemize}
\item Øget sorteringshastighed for højere udbytte
\item Reducere variation i de isolerede øer
\item Reducere omkostningerne
\item Sikre bedre dokumentation
\end{itemize} \fxnote{Reference til SG powerpoint? Specifikt nok?}

En automatiseret løsning vil potentielt åbne op for nye muligheder indenfor anvendelse af langerhanske øer. Der forskes bl.a. i transplantation af langerhanske øer, som et led i behandling af type 1 diabetes. Resultater fra \fxnote{reference} viser, at 44 \% af modtagerne af denne type behandling var insulin uafhængige 3 år efter transplantation.  (mangler henvisninger til videnskabelige artikler)


 

%\textbf{Noter:}
%
%kilder om transplantation af øer
%
%
% 
% 
% find videnskabelige artikler der bekræfter at denne metode med at skyde langerhanske øer ind i sukkersyge mus/rotter virker og det er derfor dette projekt er mega relevant og pisse godt
% 
% derfor
% 
% - videnskabelig artikel over sorteringsprocessen
% 
% - videnskabelig artikel over at metoden virker, evt. hvorfor den ikke er mere brugt(sorteringsprocessen er for langsommelig)

