\chapter{Indledning}
Insulin er et essentielt hormon til regulering af glukose i blodet. Produktionen af insulin foregår i pancreas, nærmere bestemt i de Langerhanske øer. Et fald i insulin produktionen grundet nedsat funktion af de Langerhanske øer kan medføre den livstruende sygdom \textit{diabetes mellitus (type 1)}. Prævalensen af diabetes mellitus er i Danmark på 320.545, hvor omkring 10 \% lider af type 1 diabetes. \fxnote{Ref til: http://www.diabetes.dk/presse/diabetes-i-tal/diabetes-i-danmark.aspx}




%\section*{Læsevejledning}
%Skal nok flyttes op i forord

\section{Baggrund}
Til at undersøge sygdommen nærmere og øge forståelsen for de mekanismer der styre insulin reguleringen i kroppen laves der videnskabelige forsøg med Langerhanske øer.

De øer der anvendes til videnskabelige forsøg stammer typisk fra mus eller rotter. Sortering- og isoleringsprocessen foregår over 3 faser, nærmere bestemt pancreas perfusion, pancreas digestion og til sidst ø isolering. Pancreas perfusionen foregår ved at man operativt fjerner pancreas, hvorefter vævet opløses vha. enzymet collagenase. Enzymet sprøjtes ind i pancreas, hvor en nedbrydning af vævet igangsættes. Collagenase har den egenskab, at den ikke nedbryder de Langerhanske øer. (mangler måske en kilde på hvorfor dette ikke sker?). I anden fase sker pancreas digestionen .. (mangler) \fxnote{Få protokol af SG} (Kort beskrivelse af faser: pancreas perfusion, pancreas digestion, islet purification).

I sidste fase sker selve ø isoleringen. Der findes en række forskellige metoder til dette, hvor den mest anvendte metode foregår ved manuel isolering af øerne fra en petriskål vha. et mikroskop. Der findes automatiserede isoleringsmetoder bygget på forskellige teknikker herunder gradient baseret centrifugering. Fælles for de anvendte teknikker til  isolering af øerne er, at der er stor risiko for skade på øerne.

Denne manuelle metode har yderligere ulemper i det, den både er besværlig og tidskrævende. Herudover kan der være stor variation i kvaliteten af isoleringen, da der kan være forskel på den enkelte operatørs håndtering af øerne. 

Derfor ønskes der en automatiseret metode til isolering af langerhanske øer, som minimerer risikoen for skader på øerne. En automatiseret metode vil kunne give følgende muligheder: 

\begin{itemize}
\item Øget sorteringshastighed for højere udbytte
\item Reducere variation i de isolerede øer
\item Reducere omkostningerne
\item Sikre bedre dokumentation
\end{itemize} \fxnote{Reference til SG powerpoint? Specifikt nok?}

En automatiseret løsning vil potentielt åbne op for nye muligheder indenfor anvendelse af langerhanske øer. Der er ... (mangler henvisninger til videnskabelige artikler)

Tidligere er der udviklet en prototype til automatiseret ø isolation, der fungerede ved hjælp af kamera detektion og en pipette som sugede den detekterede ø op. Prototypen var dog ikke præcis nok, primært pga. bevægelse i væsken. 
 

\textbf{Noter:}

kilder om transplantation af øer


 
 
 find videnskabelige artikler der bekræfter at denne metode med at skyde langerhanske øer ind i sukkersyge mus/rotter virker og det er derfor dette projekt er mega relevant og pisse godt
 
 derfor
 
 - videnskabelig artikel over sorteringsprocessen
 
 - videnskabelig artikel over at metoden virker, evt. hvorfor den ikke er mere brugt(sorteringsprocessen er for langsommelig)





\section{Problemformulering}
Målet med projektet er, at udvikle et \textit{Proof of Concept} system til isolering af langerhanske øer.

Metoden til isolering baseres på kamera detektion og mekanisk isolering. Det udviklede system skal altså kunne detektere langerhanske øer vha. billedprocessering for herefter at  isolere øerne fra resten af opløsningen. 

Herudover skal en cost-benefit analyse være med til at belyse hvilke økonomiske og kvalitative fordele det udviklede system vil have i sorteringsprocessen.

\newpage
\section{Afgrænsning}
Til at afgrænse projektet anvendes \textbf{M}o\textbf{SC}o\textbf{W} modellen, som beskriver hvilke dele projektet skal (\textbf{M}ust), bør (\textbf{S}houd), kan (\textbf{C}ould og ikke (\textbf{W}on't) indeholde. MoSCoW modellen (figur: \ref{fig:moscow}) viser hvordan de enkelte krav og dele af projektet er prioriteret. 


\begin{figure}[H]
	\centering
	\includegraphics[width=1\textwidth]{billeder/MoSCoW-crop.pdf}
	\caption{MoSCoW}
	\label{fig:moscow}
\end{figure}

De krav systemet skal opfylde er bl.a. at kunne detektere langerhanske øer vha. billedprocessering, samt sortere øerne ved detektion. Herudover skal systemet kunne gemme data omkring de sorterede øer, herunder størrelse og cirkularitet. Desuden består Projektet af en cost-benefit analyse der beskriver hvilke økonomiske fordele og ulemper det automatiserede system vil have. 

Et af de vigtigere krav, der er nedprioriteret i projektet er en skånsom håndtering af langerhanske øer. Denne del vil kræve en validering af den udviklede prototype igennem funktionstest af de isolerede langerhanske øer. %Referencer til dette?

Dette projekt vil derfor i højere grad fokusere på en verificering af den udviklede prototype i form af en accepttest, som tester de funktionelle og ikke funktionelle krav. Derfor er punkterne  fordelt på \textit{MoSCoW} modellen som vist. Punkterne i \textit{must have} definerer projektets minimum krav. Det er gjort for at tilpasse arbejdsbyrden i projektet, og at projektet kommer sikkert i mål.

Da de langerhanske øer er svært tilgængelige, skal der findes en simuleringsvæske. En simuleringsvæske skal bruges til at teste systemets mekaniske funktioner. De mekaniske funktioner består i isoleringen processen af de langerhanske øer, hvilket er en elementær del af projektet. Se afsnit \ref{sec:simuleringsv} for uddybelse af simuleringsvæsken.

%Redegørelse og begrundelser for valg/fravalg
%Moscow