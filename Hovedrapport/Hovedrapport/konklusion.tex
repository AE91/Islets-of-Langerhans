\chapter{Konklusion}
Der er i dette projekt udviklet et \textit{Proof of Concept} system i samarbejde med Medicinsk Endokrinologisk Afdeling, Aarhus Universitetshospital. Systemet skal anvendes til automatisk isolering af langerhanske øer i forbindelse med forskningsprojekter. Den udviklede prototype er baseret på kamera detektion og mekanisk isolering af øerne. Projektet har vist, at det er muligt at detektere langerhanske øer vha. billedeprocessering ud fra genererede billeder. \textbf{Systemet har desuden vist sig, at være velegnet til isolering af objekterne i simuleringsvæsken.} Projektet har yderligere 
 givet et indblik i hvilke problemstillinger der skal løses inden en fungerende prototype kan implementeres i praksis, herunder hvorvidt øerne tager skade af processen. 

En cost-benefit har herudover belyst, hvilke økonomiske og kvalitative fordele den udviklede prototype vil have i sorteringsprocessen. Resultaterne af denne analyse giver incitament til, at arbejde videre en automatiseret løsning til isolering af øerne. 


\textbf{Noter fra styrk projektarbejdet med indhold til konklusion:}

Den overordnede målsætning

Afgrænsning

Vigtigste resultater

De vigtigste konklusioner

\textbf{Problemformulering:}

På Medicinsk Endokrinologisk Afdeling (MEA), Aarhus Universitetshospital, har man tidligere arbejdet hen imod en automatiseret løsning til isolering af langerhanske øer. En automatiseret løsning vil primært bidrage til forskningen der foretages på afdelingen. Tidligere blev der udviklet en prototype til automatiseret ø isolation, der fungerede ved hjælp af kamera detektion. Ved detektion sugede en pipette den detekterede ø op fra en petriskål. Pipetten blev flyttet i X,Y og Z retning vha. en mekanisk arm. Prototypen var dog ikke præcis nok, primært pga. bevægelse i væsken, og projektet blev derfor stoppet.

Målet med dette projekt er, i samarbejde med Medicinsk Endokrinologisk Afdeling, at udvikle et \textit{Proof of Concept} system til isolering af langerhanske øer. Systemet bygger videre på de erfaringer man gjorde sig ved den tidligere prototype. Metoden til isolering baseres fortsat på kamera detektion, mens den mekaniske isolering i stedet erstattes med et ventilsystem. Det udviklede system skal altså kunne detektere langerhanske øer vha. billedprocessering for herefter, at isolere øerne fra resten af opløsningen. 

Herudover skal en cost-benefit analyse være med til, at belyse hvilke økonomiske og kvalitative fordele det udviklede system vil have sammenlignet med den anvendte manuelle metode.
husk at problemformuleringenspunkter skal kunne afkrydses her nede