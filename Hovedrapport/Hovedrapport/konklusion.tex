\chapter{Konklusion}
Der er i dette projekt udviklet et \textit{Proof of Concept} system i samarbejde med Medicinsk Endokrinologisk Afdeling, Aarhus Universitetshospital. Systemet skal anvendes til automatisk isolering af langerhanske øer i forbindelse med forskningsprojekter. Den udviklede prototype er baseret på kamera detektion og mekanisk isolering af øerne. Projektet har vist, at det er muligt at detektere langerhanske øer vha. billedeprocessering ud fra genererede billeder. Det lykkedes ikke i projektet, at isolere objekterne i simuleringsvæsken. Projektet har yderligere givet et indblik i hvilke problemstillinger der skal løses inden en fungerende prototype kan implementeres i praksis, herunder hvorvidt øerne tager skade af processen. 

En cost-benefit analyse har herudover belyst, hvilke økonomiske og kvalitative fordele den udviklede prototype vil have i sorteringsprocessen. Resultaterne af denne analyse giver incitament til, at arbejde videre med en automatiseret løsning til isolering af øerne. 

