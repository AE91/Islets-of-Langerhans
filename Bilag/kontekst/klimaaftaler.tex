\chapter{Klimaaftaler}

I de senere år er klarheden om klimaproblemerne vokset i det meste af verden. Klimaet står derfor højt på den internationale politiske dagsorden. Der er opstillet nogle klare aftaler indenfor klimaområdet både globalt og internationalt. Landene forpligter sig til at følge forskellige mål indenfor nedbringning af \ce{CO2}.

Den første globale miljøkonference fandt sted i Stockholm i 1972. Her blev miljøet for første gang sat på dagsordenen, men modsætningerne mellem miljø og udvikling trådte tydeligt frem og ulandene og ilandene kunne ikke opnå enighed. Først tyve år senere forsøgte FN igen. Denne gang i Rio de Janeiro i 1992. Konferencen førte til en aftale om begrænsning af den globale udledning af drivhusgasser. Aftalen blev i sin
tid underskrevet af 183 nationer, men er i dag oppe på 194 medlemslande.

Den nok mest kendte aftale om at beskytte jordens klima er Kyoto-aftalen, hvilken der vil blive redegjort for i det følgende.

\section{Kyoto-aftalen}

Kyoto-aftalen blev indgået den 11. december 1997 i Kyoto, Japan. Aftalen indebærer, at de globale udslip af drivhusgasser skal reduceres med 5,2 \% i forhold til 1990-niveau frem til perioden 2008–2012. Protokollen indebærer bl.a., at EU skal reducere sine udslip med 8 \%, USA med 7 \%, og Japan med 6 \%, mens det det for Rusland, Kina og Indien er 0 \%. Traktaten tilladte stigninger på 8 \% for Australien og 10 \% for Island.

Kyoto-aftalen trådte i kraft, da Rusland ratificerede aftalen. I november/december 2005 afholdtes den første konference mellem Kyoto-aftalens parter (COPMOP1) i Montreal i Canada. Samtidig fortsatte forhandlinger med de lande, der har valgt at gå udenfor Kyoto-aftalen. USA er det land, der udleder flest drivhusgasser, og netop USA har afvist at deltage i Kyoto-samarbejdet. Indien er heller ikke med i samarbejdet. Kina er på grund af sin status som udviklingsland ikke bundet til at reducere sin udledning af drivhusgas på trods af, at landet mængdemæssigt har den største udledning i verden. I 2008, har 183 lande ratificeret protokollen.

Kyoto-protokollen er en protokol til FN's rammekonvention om klimaændringer (UNFCCC eller FCCC), en international miljø-traktat produceret på De Forenede Nationers Konference om Miljø og Udvikling (UNCED), uformelt kaldet Earth Summit, der blev afholdt i Rio de Janeiro, Brasilien, 3-14 juni 1992. Traktaten har til formål at opnå "en stabilisering af koncentrationerne af drivhusgasser i atmosfæren på et niveau, som kan forhindre farlig menneskeskabt påvirkning af klimasystemet.". Kyoto-protokollen fastlægger juridisk bindende forpligtelser til reduktion af seks drivhusgasser (kuldioxid, Metan, lattergas, svovlhexafluorid, hydrofluorcarboner og perfluorcarboner) produceret af UNFCCC-tilsluttede nationer, såvel som generelle forpligtelser for alle medlemslande.

\subsection{Kvoter}

Kyoto-protokollen omfatter defineret "fleksible mekanismer" som emissionshandel, Clean Development Mechanism og Joint Implementation, der tillader UNFCCC-lande til at opfylde deres forpligtelser ved at købe emissionsreduktioner fra andre steder, gennem finansielle udvekslinger, f.eks. projekter, som mindsker emissioner i ikke-UNFCCC-lande, fra andre UNFCCC-lande eller fra UNFCCC-lande med overskydende kvoter. I praksis betyder dette, at ikke-UNFCCC-økonomier ingen restriktioner har, men har økonomiske incitamenter til at udvikle reduktioner af drivhusgasemissioner ved at at modtage "Carbon Credits", som dernæst kan sælges til UNFCCC-købere, og derved fremme bæredygtig udvikling.

Dertil kommer, at de fleksible mekanismer tillader UNFCCC-lande lande med effektive, lave DHG-emitterende industrier, og høje gældende miljøstandarder at købe kulstoftilgodehavender på verdensmarkedet i stedet for at reducere udledningen af drivhusgasser på hjemmemarkedet. UNFCCC-lande vil typisk ønske at erhverve Carbon Credits så billigt som muligt, mens ikke-UNFCCC-lande ønsker at maksimere værdien af kulstoftilgodehavender genereret fra deres hjemlige Greenhouse Gas Projects.

\subsection{Kritik af aftalen}

Der er blevet kritiseret at lande som Indien, Brasilien og Kina (med Kina som største udleder af CO2 i 2010) slet ikke er forpligtet af protokollen. Dette skyldes at de tilbage ved vedtagelsen i 1997 ikke blev anset for industrilande men derimod U-lande. Det er her diskuteret om dette er et tidssvarende syn i dag, men landende har ikke vist nogen interesse i at lade sig forpligte. Et andet eksempel på kritik er udsagnet om, at de enkelte lande prøver at undgå regelsættet, ved at fjerne deres CO2 udslip ved at flytte det til fattige lande, eller køber kvoter af de fattige lande så der reelt set ikke sker en udvikling der vil påvirke verdensklima i en positiv retning.

En anden kritik er at f.eks. Maersk ikke bliver talt med i Danmarks regnskab da det ses som skibsfart, der ikke kun gavner Danmark.

\section{Ny global aftale}

På COP17 i Durban i 2011 enedes parterne om, at man i 2015 skal vedtage en ny global klimaaftale med reduktionsforpligtelser for alle parter med ikrafttrædelse senest i 2020. Denne aftale skal have form af en protokol, et andet retligt instrument eller et ”agreed outcome with legal force under the Convention applicable to all Parties". 

På COP18 i Doha i 2012 fik man vedtaget et arbejdsprogram for forhandlingerne af denne aftale frem mod 2015. Dette arbejdsprogram blev yderligere konkretiseret på COP19 i Warszawa i 2013, hvor man bl.a. enedes om, hvornår i processen frem mod 2015 parterne bør fremlægge deres bidrag til en ny aftale.

\subsection{Forpligtelser}

Et af de store spørgsmål ved udformningen af en global klimaaftale, er hvordan Klimakonventionens principper om "fælles men differentieret ansvar" (Common But Differentiated Responsibility - CBDR) og "respektive kapaciteter" (Respective Capacities - RC), skal anvendes i en aftale. Herudover skal man blive enige om, hvordan reduktionsindsatser for alle parter (”applicable to all”) skal forstås. Ulandene er i klimaforhandlingerne meget optagede af, at ilandene som følge af historisk ansvar for udledninger af drivhusgasser må gå forrest. 

Ulandene har endvidere argumenteret for, at da forhandlingerne om en global klimaaftale er nedsat under Klimakonventionen, skal konventionens principper om CBDR, RC og ”equity” (lige adgang til bæredygtig udvikling for alle parter) udgøre hjørnestenene i en fremtidig aftale. Ilandene, herunder EU, har heroverfor anført, at man ikke er ude på at ændre Klimakonventionens principper samt at mandatet fra COP17 fastslår, at en ny aftale skal indeholde reduktionsforpligtelser for alle parter. EU og Danmark ønsker, at Konventionens principper i en ny aftale fortolkes på en dynamisk måde, der afspejler parternes nutidige formåen og ikke et verdensbillede anno 1992.

Et kompromis kan blive, at en global klimaaftale vil indeholde et spektrum af differentierede forpligtelser (”spectrum of commitments”). Dette spektrum skal tage højde for parternes nuværende formåen og nationale omstændigheder, og ikke kun skele til parametre så som historisk ansvar for klimaforandringerne. Under en sådan model vil de rigeste lande fortsat skulle gå forrest og bære en stor del af ansvaret for at reducere udledningen af drivhusgasser.

Alle parter skal dog som udgangspunkt forpligte sig i et givet omfang, også de store vækstøkonomier, som står for en stigende andel af den nuværende udledning af drivhusgasser. Endvidere bør en global aftale udfærdiges så tilpas fleksibel, at det er muligt at justere parternes reduktionsforpligtelser hen ad vejen uden at skulle genforhandle selve aftalen. Derved kan undgås, at man i en ny aftale fastlåser reduktionsindsatsen på et niveau, der er utilstrækkeligt ift. at nå målsætningen om at holde den globale temperaturstigning på under 2 grader.

Parterne er enige om, at parternes bidrag til en ny aftale skal fastsættes nationalt (”nationally determined”). Det betyder konkret, at parterne selv fastlægger de målsætninger, de forventer at kunne bidrage med i en ny aftale. Danmark og EU ønsker derfor, at en ny aftale indeholder et robust rapporteringsregime for måling, rapportering og verificering af parternes bidrag, så det derved sikres, at disse er reelle og tilstrækkeligt ambitiøse.

\subsection{Forhandlingerne efter COP19}

På trods af at forhandlingerne om en ny global klimaaftale i 2013 generelt var konstruktive og mere konkrete end i 2012, så afslørede COP19 i Warszawa en grundlæggende uenighed om, hvorvidt mandatet fra COP17 i Durban skal forstås sådan, at alle parter skal påtage sig juridisk bindene mål (”commitments”) i en ny klimaaftale. En række ulande søgte på COP19 vedholdende at genintroducere den skarpe opdeling (”firewall”) mellem i- og ulande, som med mandatet fra COP17 blev nedbrudt.

Efter lange forhandlinger lykkedes det parterne at nå et til et kompromis, hvor man i beslutningsteksten fra COP19 om en køreplan frem til 2015 enedes om at erstatte ordet ”commitments” med ”contributions”. Samtidig undlod man bevidst at tage stilling til bidragenes juridiske status. Spørgsmålet om, hvilke parter, der skal påtage sig juridiske forpligtelser blev dermed sparket til hjørne. Parterne opfordredes til ”i god tid” inden COP21 i Paris - i første kvartal af 2015 for de parter, ”der er klar hertil” – at fremlægge bidrag til en ny aftale i form af reduktionstilsagn. Herudover enedes parterne, om, at man på COP20 i Lima i december 2014 skal identificere den information (”up front information”), som parterne skal fremlægge sammen med disse bidrag for derved at skabe mere klarhed om disses reelle betydning for parternes udledninger.